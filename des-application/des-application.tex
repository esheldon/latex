%\documentclass[12pt]{article}
\documentclass[12pt]{letter}

\begin{document}


\hfill August 21,  2007
\newline

Dear Dr. Peoples,

I am writing to apply for membership (participant status) in the Dark Energy
Survey collaboration, under the terms of the Policy on DES Participation for
Students and Postdocs. I began working on the DES in late 2003 while I was a
postdoctoral research associate in the Kavli Institute for Cosmological Physics
at the University of Chicago, under the supervision of Josh Frieman. I have
continued working on DES project infrastructure since moving to a postdoctoral
position in the astrophysics group in the New York University Physics
Department in 2005.  Over the last three and half years, I have made a number
of contributions to DES and am committed to continuing to do so in the future.

My contributions to DES so far include:

\begin{quotation}
(i) I carried out simulations to predict the shear sensitivity of the DES for
weak lensing. This involved degrading deep, high-resolution images from the HST
GOODs to the noise level and seeing of the DES and measuring the resulting
shear sensitivity. This enabled the collaboration to make forecasts for the
statistical errors on the weak lensing power spectrum and derive resulting
constraints on the dark energy equation of state. It also quantifies the
expected gains/losses in weak lensing if the median delivered PSF is greater or
smaller than the current median from the Mosaic II camera. The results of this
study were included in the BIRP proposal and in the later joint proposal to DOE
and NSF. 
\newline

(ii) I carried out simulations of DES images with varying pixel scale, in order
to test the effects of finite pixel scale on the shear measurements.  This work
was written up as a short report and circulated within the collaboration. The
conclusion was that a pixel scale of 0.27" should not lead to substantial
degradation of shape measurements given the range of expected seeing conditions
for the survey.
\newline

(iii) I performed a shapelet analysis of a volume-limited SDSS galaxy catalog
and used this to derive the shapelet distribution for well-resolved galaxies.
The shape of any galaxy can be decomposed into a sum of shapelets with varying
coefficients.  This catalog was the basis for the simulated galaxy images
placed into the image-level DES simulations used for the DES DM data
challenges.
\newline

(iv) I have worked with Mike Jarvis to port his shapelet and star-galaxy
separation codes to work on DES images. This will be the primary code used for
PSF characterization in the DES DM pipeline.
\end{quotation}

In the future, I plan to continue making contributions both to DES DM and to
the weak lensing analysis pipelines. Specifically:

\begin{quotation}
(v) Mike Jarvis and I will create a galaxy shapelet code that will work in the
DES DM system.  This will involve creating a new code to work on the entire
list of images available for each object detected in the coadd.  The PSF
information for each exposure will be integrated into the measurement.  The DM
requirements for this are very strict, because we require random access to all
image data, image metadata, coadd object catalogs, and PSF information.  This
information must be efficiently accessible.  These requirements will be a
significant driver for the DES DM infrastructure from a data retrieval point of
view, and I will work directly with the DM team to implement what we need.
\newline

(vi) I will develop codes for testing the image-level simulations and for
testing science post-processing codes on the image-level sims, recovering
physical input, etc.  This work is important to provide an end-to-end test of
the DES DM and science analysis pipelines in the context of the data
challenges.  I plan to test catalog-level simulations in a similar way.
\newline

(vii) I will develop generic cross-correlation code for use in DES DM or in the
weak lensing analysis pipeline. This code will be used for measuring
object-shear cross-correlatios and galaxy counts as a function of separation
and other variables around arbitrary points.  This code is mostly in place but
will need to work in the DM or science pipeline environment, perhaps on the
grid. These cross-correlation measurements will provide some of the key results
from DES WL.
\end{quotation}

On the scientific side, I will bring experience with lensing measurements,
object-galaxy cross correlations, photometric redshifts, and cluster finding,
especially from the SDSS. 

My primary science goal is to use cross-correlation lensing to determine the
mass-observable relation for DES/SPT clusters selected in various ways.  This
measurement requires essentially all primary DES/SPT technical goals to be met
and the success of both projects hinges on it.

I will also perform cross correlations between clusters and the overall galaxy
population. We have shown in the SDSS that this is a powerful tool for learning
about M/L profiles, cluster galaxy populations, bias, and cosmology.  Similar
studies will be done with LRGs and other objects easily selected and for which
good photometric or spectroscopic redshifts are available.  

My experience with cluster finding and photoz-s should help bridge the gap
between the photo-z, cluster, and weak lensing working groups.

I expect to be involved at some level in all DES lensing projects, since I
expect Mike Jarvis and I will be primarily responsible for producing the galaxy
shape measurement catalogs that will be used in all DES lensing studies. We
believe the shapelet measurements may also provide the best galaxy ``model"
magnitude estimates for various other science goals (e.g., for determining
galaxy colors for photo-z estimates), and we will explore those applications as
well.
\newline

{\noindent Erin Sheldon}


\end{document}
