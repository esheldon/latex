\documentclass[preprint]{aastex}

\begin{document}

%\title{test}

\section{Shear Sensitivity}

The figure of merit for lensing is the shear sensitivity.  For no noise, this
is simply the standard deviation of source galaxy ellipticity (or "shape
noise") over root N, divided by 2 to convert ellipticity to shear.  In the real
world there is noise, and this must be taken into account.  Hu and Jain put
this effect into a larger assumed shape noise.  Also, small objects (relative
to the PSF) contribute less signal, and this is just as important as the noise.

I used Huan's CFH12k data to estimate the shear sensitivity for the magnitude
limits 24.1 and 23.7, and a range of seeing.  I ran his images through my shape
measurement code, which calculates the shapes and their noise, as well as their
size.  Given these data, I can calculate the shear sensitity.  Usually Fisher
matrix code actually takes in the shape noise and the number density, so I
converted the shear sensitivity to an effective number density assuming the
shape noise is 0.32.  Then the S/N just goes as sqrt(effective area).

Figure \ref{fig:weight} shows the lensing weight as a function of magnitude
(IAB), which drops rapidly after IAB = 22.5. The numbers are in table
\ref{tab:weight}. Figure \ref{fig:sens} shows the shear sensitity vs. seeing,
and the effective number density vs. seeing, for each of the mag limits 23.7
and 24.1 and a survey of 5000 square degrees.  Note this is an over-estimate of
the sensitivity for 23.7 because this is a pretty deep image so the noise is a
bit underestimated.

As you can see, the effective number density drops rapidly with seeing.
This is also summarized in table \ref{tab:sens}.

I took a very crude look at how we would do with the equation of state:

Hu and Jain calculated their stuff for a feducial survey of 70 galaxies per
square arcmin and 4000 square degrees.  They also assumed a large shape noise
of 0.6 in order to account for measurement error.  We would do a little better
than half (0.63) as well in shear sensitivity compared to their survey.  On the
other hand, they also assumed a mean source redshift of 1.5, whereas we will
have more like 0.7 as I recall.  This means the signal will be smaller for our
survey by roughly a factor of 2.  So the signal to noise will be something like
a third as high as their estimates.  They showed that you could expect
something like 0.02 in a constant w for their survey under the best possible
conditions. So we are looking at 0.06 in w for 1" seeing.  This would drop
to something like 0.04 or so for 0.7" seeing.

This is not exactly right because the lever arm in z is smaller for our survey,
and thus the constraints will be weaker since there is less time over which to
see the effect of lambda on the angular diameter distances.  In other words, we
should do the full calculation to get it right, but you get the general
picture.

Note also that the weights vs. mag. must also be applied to any estimate of the
effective source redshift distribution for lensing.

\epsscale{0.8}

\begin{figure}
\plotone{huan_weights_vs_mag.eps} \figcaption{ The relative weight as a
function of magnitude, accounting for shape measurement uncertainties and
dilution.
\label{fig:weight} }
\end{figure}


\begin{figure}
\plotone{huan_sensitivity.eps} \figcaption{ The shear sensitivity (top) and
effective density (bottom) as a function of magnitude cut and seeing.  The
feducial survey is 5000 square degrees. This is an overestimate of the
sensitivity for IAB $< 23.7$ since the extra noise was not accounted for.
\label{fig:sens} }
\end{figure}

\begin{deluxetable}{ccccc}
  \tabletypesize{\small}
  \tablecaption{Relative Weight\label{tab:weight}}
  \tablewidth{0pt}
  \tablehead{
    \colhead{$~~~~~~$IAB$~~~~~~$} &
    \colhead{$~~~~~~$weight$~~~~~~$} 
  }
  \
  \startdata
  18.19 & 1.000 \\ 
  18.46 & 1.000 \\ 
  18.69 & 1.000 \\ 
  18.93 & 1.000 \\ 
  19.19 & 1.000 \\ 
  19.43 & 1.000 \\ 
  19.68 & 0.999 \\ 
  19.93 & 0.999 \\ 
  20.18 & 0.998 \\ 
  20.43 & 0.997 \\ 
  20.68 & 0.996 \\ 
  20.93 & 0.994 \\ 
  21.18 & 0.992 \\ 
  21.43 & 0.988 \\ 
  21.68 & 0.982 \\ 
  21.93 & 0.972 \\ 
  22.18 & 0.958 \\ 
  22.43 & 0.936 \\ 
  22.68 & 0.904 \\ 
  22.93 & 0.859 \\ 
  23.18 & 0.799 \\ 
  23.43 & 0.716 \\ 
  23.63 & 0.648
  \enddata
  \tablecomments{Relative Weight as a function of IAB mag, accounting for measurement error and dilution.  The seeing for this data is 0.627\arcsec.}
\end{deluxetable}

\begin{deluxetable}{ccccc}
  \tabletypesize{\small}
  \tablecaption{Effective number density\label{tab:sens}}
  \tablewidth{0pt}
  \tablehead{
    \colhead{seeing} &
    \colhead{$n$ (IAB $<$ 23.7)} &
    \colhead{$n$ (IAB $<$ 24.1)} \\
    \colhead{\arcsec} & 
    \colhead{arcmin$^{-2}$} &
    \colhead{arcmin$^{-2}$} 
  }
  \
  \startdata
  0.627 & 16.072 & 19.668 \\ 
  0.650 & 15.750 & 18.902 \\ 
  0.675 & 15.621 & 18.281 \\ 
  0.700 & 14.902 & 17.634 \\ 
  0.725 & 14.136 & 16.800 \\ 
  0.750 & 13.305 & 15.802 \\ 
  0.775 & 12.322 & 14.614 \\ 
  0.800 & 11.461 & 13.646 \\ 
  0.825 & 10.461 & 12.475 \\ 
  0.850 & 9.7207 & 11.459 \\ 
  0.875 & 9.0448 & 10.245 \\ 
  0.900 & 8.1732 & 9.4532 \\ 
  0.925 & 7.2912 & 8.6110 \\ 
  0.950 & 6.8471 & 7.7473 \\ 
  0.975 & 6.2774 & 7.1323 \\ 
  1.000 & 5.6600 & 6.4134 \\ 
  1.025 & 5.1051 & 5.7860 \\ 
  1.050 & 4.7651 & 5.4556 \\ 
  1.075 & 4.3364 & 4.7119 \\ 
  1.100 & 3.8980 & 4.2987 \\ 
  1.125 & 3.5550 & 3.9998 \\ 
  1.150 & 3.3043 & 3.5393 \\ 
  1.175 & 3.0197 & 3.2472 \\ 
  1.200 & 2.7537 & 2.8881
  \enddata
  \tablecomments{Effective number density $n$ as a function of seeing and magnitude cut. Defined such that shear sensitivity $\equiv 0.32/\sqrt{n*area}$}
\end{deluxetable}


\end{document}
