\begin{deluxetable}{lcc}
    \tabletypesize{\small}
    \tablecaption{Projected Computing Purchases\label{table:computing}}
    \tablewidth{0pt}
    \tablehead{
        \multicolumn{1}{l}{Fiscal Year} &
        \colhead{Compute Servers with 2 GPUs}   & 
        \colhead{Total Storage} \\
        &
        &
        [TB]
    }
    \startdata
2013 & 4 & 32 \\
2014 & 4 & 32 \\
2015 & 4 & 32 \\
2016 & 4 & 32 \\
2017 & 4 & 32 \\
\hline
\relax\\[-1.7ex]
Total in 5 years & 10 & 160 TB \\\\[-2.7ex]
\enddata

    \tablecomments{The number of compute nodes purchased is based on the
    assumption that each node and GPU (Intel 12 cores, 32GB ram, two nvidia
    2050 GPUs) would stay at the performance level of a node purchased in 2012.
    It is not clear how the performance per GPU will increase over time so this
    table is conservative.  Disk is likely to get cheaper, so the 8TB per
    machine could be expanded at fixed cost. Power, cooling and maintenance
    will be provided at no extra cost to this experiment, but overhead 
    and slight escalation are included in the budget.}

    \end{deluxetable}
    
