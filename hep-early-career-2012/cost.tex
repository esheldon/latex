\begin{deluxetable}{lcccc}
    \tabletypesize{\small}
    \tablecaption{Projected Computing Purchases\label{table:computing}}
    \tablewidth{0pt}
    \tablehead{
        \multicolumn{1}{l}{Fiscal Year} &
        \colhead{Compute Servers with GPUs}   & 
        \colhead{Total Storage} &
        \colhead{Cost in \$} \\
        &
        &
        [TB] &
        2012 Equivalent &
    }
    \startdata
2013 & 2 & 26 & 17600 \\
2014 & 2 & 26 & 17600 \\
2015 & 2 & 26 & 17600 \\
2016 & 2 & 26 & 17600 \\
2017 & 2 & 26 & 17600 \\
\hline
\relax\\[-1.7ex]
Total in 5 years & 10 & 130 TB & 88000 \\\\[-2.7ex]
\enddata

    \tablecomments{The number of compute nodes purchased is based on the
    assumption that each node and GPU (Intel 12 cores, 32GB ram, two nvidia
    2050 GPUS) would stay at the performance level of a node purchased in 2012.
    It is not clear how the performance per GPU will increase over time so this
    table is conservative.  Disk is likely to get cheaper, so 13TB per machine
    at this price is a lower limit. Power, cooling and maintenance will be
    provided at no extra cost to this experiment, but overhead of \overhead\
    must be additionally included.}

    \end{deluxetable}
    
