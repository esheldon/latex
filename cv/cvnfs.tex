\documentclass[10pt]{article}
\usepackage{titlesec}
\usepackage{tabularx}
%\usepackage{times}
\usepackage[paper=letterpaper,
	left=1in,
	right=1in,
	top=1in,
	bottom=1in
]{geometry}

\setlength{\parindent}{0pt}

\renewcommand\thesection{\Alph{section}}
\titlespacing*{\section}{0.5em}{0.5em}{0.5em}
\newcolumntype{R}{>{\raggedleft\arraybackslash}X}

\begin{document}

\begin{center}
{\huge \sc Erin Sheldon}\\
\end{center}

%Physics \& Astrophysics 209\\
%Physics Department, Stanford University\\
%382 Via Pueblo Rd\\
%Stanford, CA 94305-4060\\
%email: \verb+beckermr@stanford.edu+\\
%web: \verb+http://www.stanford.edu/~beckermr+\\

{\large \bf A. Professional Preparation}
\vspace{-0.5em}
\begin{table}[h]
\begin{tabular}{lll}
University of Missouri-Columbia, Columbia, MO & Physics and Mathematics & B.S. 1997\\
University of Michigan-Ann Arbor, Ann Arbor, MI &  Physics &  Ph.D., 2002\\
The University of Chicago, Chicago, IL & KICP Fellow & 2002-2005\\
New York University, New York, NY & Postdoc CCPP & 2005-2008\\
\end{tabular}
\end{table}
%\section{Professional Preparation}
%A list of the individualÕs undergraduate and graduate education and postdoctoral training as indicated
%below:
%Undergraduate Institution(s) Major Degree \& Year
%Graduate Institution(s) Major Degree \& Year
%Postdoctoral Institution(s) Area Inclusive Dates (years)

{\large \bf B. Appointments}
\vspace{-0.5em}
\begin{table}[h]
\begin{tabularx}{\textwidth}{lR}
\textsc{Physicist} & 2012 -- present\\
\textsc{Associate Physicist} & 2008 -- 2010\\
\textsc{Assistant Physicist} & 2010 -- 2012\\
Department of Physics, Brookhaven National Laboratory \\
\end{tabularx}
\end{table}
%A list, in reverse chronological order, of all the individualÕs academic/professional appointments beginning
%with the current appointment.
%If the proposer has a website address readily available, that information should be included in the citation, as stated
%above. It is not NSF's intent, however, to place an undue burden on proposers to search for the URL of every
%referenced publication. Therefore, inclusion of a website address is optional. A proposal that includes reference
%citation(s) that do not specify a URL address is not considered to be in violation of NSF proposal preparation
%guidelines and the proposal will still be reviewed.

{\large \bf C. Products}\\
\vspace{-0.75em}\\
{\bf Five Closely Related Products:}
\begin{enumerate}

    \item \textit{An implementation of Bayesian lensing shear measurement}\\
        {\bf E. S. Sheldon}, {\bf 2014}, MNRAS, 444, 115

    \item \textit{Cross-correlation Weak Lensing of SDSS Galaxy Clusters. III. Mass-to-Light Ratios}\\
        {\bf E. S. Sheldon}, \em{et al.}, {\bf 2009}, ApJ, 703, 2232

    \item \textit{Cross-correlation Weak Lensing of SDSS Galaxy Clusters. I. Mesurements}\\
        {\bf E. S. Sheldon}, \em{et al.}, {\bf 2009}, ApJ, 703, 2217

    \item \textit{The Galaxy-Mass Correlation Function Measured from Weak Lensing in the Sloan Digital Sky Survey}\\
        {\bf E. S. Sheldon}, \em{et al.}, {\bf 2004}, ApJ, 127, 2544

    \item \textit{Weak-Lensing Measurements of 42 SDSS/RASS Galaxy Clusters}\\
        {\bf E. S. Sheldon}, \em{et al.}, {\bf 2001}, ApJ, 554, 881

\end{enumerate}
{\bf Five Additional Significant Products:}
\vspace{-0.5em}
\begin{enumerate}

    \item \textit{Cosmological constraints from the large-scale weak lensing of SDSS MaxBCG clusters}\\
        Ying Zu, David H. Weinberg,Eduardo Rozo, {\bf E. S. Sheldon}, Jeremy Tinker; Matthew Becker,
        {\bf 2014}, MNRAS, 439, 1628

    \item \textit{Cosmological Constraints from Galaxy Clustering and the Mass-to-number Ratio of Galaxy Clusters}\\
        Tinker, J.~L., {\bf Sheldon, E.~S.}, {\em et al.}, {\bf 2012}, ApJ, 745, 16

    \item \textit{Cosmological Constraints from the SDSS maxBCG Cluster Catalog}\\
        E. Rozo, R. H. Wechsler, E. S. Rykoff, J. T. Annis, 
        M. R. Becker, A. E. Evrard, J. A. Frieman, S. M. Hansen, 
        J. Hao, D. E. Johnston, B. P. Koester, T. A. McKay, 
        {\bf E. S. Sheldon}, D. H. Weinberg, {\bf 2010}, ApJ, 708, 645

\end{enumerate}

%A list of: (i) up to 5 publications most closely related to the proposed project; and (ii) up to 5 other
%significant publications, whether or not related to the proposed project. Each publication identified must
%include the names of all authors (in the same sequence in which they appear in the publication), the
%article and journal title, book title, volume number, page numbers, and year of publication. If the document
%is available electronically, the website address also should be identified.
%For unpublished manuscripts, list only those submitted or accepted for publication (along with most likely
%date of publication). Patents, copyrights and software systems developed may be substituted for
%publications.
%Additional lists of publications, invited lectures, etc., must not be included.

{\large \bf D. Synergistic Activities}
\begin{description}
\item[Scientific Collaborations:] {\bf Dark Energy Survey Collaboration:} Co-organizer of the Weak Lensing 2pt Functions Analysis Group; Member, 
Weak Lensing Working Group, Simulations Working Group; {\bf LSST Dark Energy Science Collaboration:} Full Member, Clusters Working Group, Cosmological Simulations Working Group
\item[Awards:] Nathan Sugarman Award for Excellence in Graduate Student Research, University of Chicago; GAANN Graduate Fellowship, University of Chicago; LeRoy Apker Award for Undergraduate Thesis, American Physical Society
\item[Scientific Code and Tool Development:] \textsc{CALCLENS}, parallel, adaptive code to produce weak lensing predictions for large-area sky surveys, publicly available at \verb+https://github.com/beckermr/calclens+; Co-developer of the \textsc{Blind Cosmology Challenge Simulations}, a large suite of simulations with weak lensing signals used by the Dark Energy Survey Collaboration for testing and data analysis
\end{description}

%A list of up to five examples that demonstrate the broader impact of the individualÕs professional and
%scholarly activities that focuses on the integration and transfer of knowledge as well as its creation.
%Examples could include, among others: innovations in teaching and training (e.g., development of
%curricular materials and pedagogical methods); contributions to the science of learning; development
%and/or refinement of research tools; computation methodologies, and algorithms for problem-solving;
%development of databases to support research and education; broadening the participation of groups
%underrepresented in science, mathematics, engineering and technology; and service to the scientific and
%engineering community outside of the individualÕs immediate organization.

{\large \bf E. Collaborators \& Other Affiliations}
\begin{description}
\item[Collaborators and Co-Editors:]

Tom Abel (Stanford/SLAC),
Adam Amara (ETH Zurich),
Peter Behroozi (STScI),
Bradford Benson (Univ. of Chicago \& FNAL),
Andreas Berlind (Vanderbilt),
Gary Bernstein (Univ. of Penn.),
Michael Blanton (NYU),
Lindsey Bleem (ANL),
Sarah Bridle (Univ. of Manchester),
Michael Busha (Elastica),
John Carlstrom (Univ. of Chicago),
Chihway. Chang (ETH Zurich),
Tom Crawford (Univ. of Chicago),
Carlos Cunha (Bosch), 
Chris Davis (Stanford),
Joerg Dietrich (USM/LMU),
Tim Eifler (NASA JPL/Caltech),
Brandom Erickson (Northrop Grumman),
August Evrard (Univ. of Michigan-Ann Arbor), 
Oliver Friedrich (USM/LMU),
David Gerdes (Univ. of Michigan-Ann Arbor), 
Daniel Gruen (USM/LMU \& MPE),
Jiangang Hao (ETS \& Fermilab), 
Andrew Hearin (Yale),
F. Will High (Netflix), 
Gill Holder (McGill),
Bhuvnesh Jain (Univ. of Penn.),
Michael Jarvis (Univ. of Penn.),
Ryan Keisler (Stanford),
Andrey Kravtsov (Univ. of Chicago),
Benjamin P. Koester (Univ. of Michigan),
Boris Leistedt (UCL),
Niall MacCrann (Univ. of Manchester),
Phil Marshall (SLAC),
Timothy McKay (Univ. of Michigan-Ann Arbor), 
Annalisa Mana (USM/LMU \& MPE),
S. Marru (Indiana Univ.),
M. Pierce (Indiana Univ.),
Andres Plazas Malagon (JPL),
Rachel Reddick (Insight Data Science Fellow),
Alexandre Refregier (ETH Zurich),
Reina Reyes (Ateneo de Manila Univ.),
Eduardo Rozo (Univ. of Arizona), 
Eli Rykoff (SLAC),
Stella Seitz (USM/LMU \& MPE),
Erin Sheldon (BNL),
Raminder Singh (Indiana Univ.),
Ramin Skibba (UC San Diego),
M. Soares-Santos (FNAL), 
Jeeseon Song (Korea Astronomy and Space Science Institute),
Kyle Story (Univ. of Chicago),
Eric Suchyta (OSU),
Molly Swanson (Harvard), 
Greg Tarle (Univ. of Michigan-Ann Arbor), 
Jeremy Tinker (NYU),
Michael Troxel (Univ. of Manchester),
Douglas Watson (Univ. of Chicago),
Risa Wechsler (Stanford/SLAC),
David Weinberg (OSU),
Andrew Zentner (Univ. of Pittsburgh),
Idit Zehavi (Case Western),
Yuanyuan Zhang (Univ. of Michigan-Ann Arbor),
Ying Zu (CMU),
Joe Zuntz (Univ. of Manchester)

\item[Graduate Advisors and Postdoctoral Sponsors:] \textit{Graduate Advisor:} Andrey~V.~Kravtsov (The University of Chicago), \textit{Postdoctoral Sponsors:} Risa~Wechsler (Stanford University \& SLAC)
\item[Thesis Advisor and Postgraduate-Scholar Sponsor:] \textit{None}, Total Number of Graduate Students: 0, Total Number of Postdoctoral Scholars: 0
\end{description}

%A list of all persons in alphabetical order (including their current
%organizational affiliations) who are currently, or who have been collaborators or co-authors with the
%individual on a project, book, article, report, abstract or paper during the 48 months preceding the 
%submission of the proposal. Also include those individuals who are currently or have been co-editors of
%a journal, compendium, or conference proceedings during the 24 months preceding the submission of
%the proposal. If there are no collaborators or co-editors to report, this should be so indicated.
%Graduate Advisors and Postdoctoral Sponsors. A list of the names of the individualÕs own graduate
%advisor(s) and principal postdoctoral sponsor(s), and their current organizational affiliations.
%Thesis Advisor and Postgraduate-Scholar Sponsor. A list of all persons (including their
%organizational affiliations), with whom the individual has had an association as thesis advisor, or with
%whom the individual has had an association within the last five years as a postgraduate-scholar sponsor. 

\end{document}
%%%%%%%%%%%%%%%%%%%%%%%%%% End CV Document %%%%%%%%%%%%%%%%%%%%%%%%%%%%%
