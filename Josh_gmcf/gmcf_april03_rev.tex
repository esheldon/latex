%\documentclass[12pt,preprint]{aastex}
%\documentclass{aastex}
%\documentclass[manuscript]{aastex}
%\documentclass[preprint2]{aastex}
%\documentclass[preprint]{aastex}

%For submission to AJ
%\documentclass{aastex}
%For astro-ph
\documentclass{emulateapj}

\slugcomment{Last revision 16-Sept-2003}

\shortauthors{Sheldon et al.}
\shorttitle{Galaxy-mass Correlations in SDSS}

\newcommand{\umag}{$u$}
\newcommand{\gmag}{$g$}
\newcommand{\rmag}{$r$}
\newcommand{\imag}{$i$}
\newcommand{\zmag}{$z$}
\newcommand{\gmr}{$g-r$}
\newcommand{\deltasig}{$\Delta \Sigma$}
\newcommand{\deltaplus}{$\Delta \Sigma_+$}
\newcommand{\deltacross}{$\Delta \Sigma_\times$}

\newcommand{\photo}{\texttt{PHOTO}}
\newcommand{\astrop}{\texttt{ASTRO}}
\newcommand{\mt}{\texttt{MT}}
\newcommand{\spectro}{\texttt{SPECTRO}}
\newcommand{\spectroone}{\texttt{SPECTRO1d}}
\newcommand{\spectrotwo}{\texttt{SPECTRO2d}}
\newcommand{\target}{\texttt{TARGET}}

\def\eone{e$_1$}
\def\etwo{e$_2$}
\newcommand{\etan}{e$_+$}
\newcommand{\erad}{e$_\times$}
\newcommand{\eclass}{\texttt{ECLASS}}
\newcommand{\eclasscut}{-0.06}
\newcommand{\gmrcut}{0.7}

\newcommand{\hrs}{$^{\mathrm h}$}
\newcommand{\minutes}{$^{\mathrm m}$}

\newcommand{\ugriz}{$u, g, r, i, z$}
\newcommand{\polarization}{polarization}

\newcommand{\wgm}{$w_{gm}$}
\newcommand{\wgg}{$w_{gg}$}
\newcommand{\wmm}{$w_{mm}$}
\newcommand{\xigg}{$\xi_{gg}$}
\newcommand{\ximm}{$\xi_{mm}$}
\newcommand{\xigm}{$\xi_{gm}$}

\newcommand{\numspec}{127,001}
\newcommand{\numrand}{1,270,010}
\newcommand{\numspectot}{278,192}
\newcommand{\numvdis}{49,024}
%\newcommand{\numsource}{10,259,949}
% hirata: 
\newcommand{\numsource}{9,020,388}
\newcommand{\numpairs}{4,776,222,720}
\newcommand{\altnumpairs}{4.8 billion}
\clearpage


\begin{document}

\title{Galaxy-mass Correlations from Weak Lensing in the SDSS}

\author{
Erin Sheldon,\altaffilmark{1,2}
David E. Johnston,\altaffilmark{2}
Ryan Scranton,\altaffilmark{2}
A. J. Connolly,\altaffilmark{3}
Joshua A. Frieman,\altaffilmark{2,4}
Timothy A. McKay,\altaffilmark{5}
Many Others
}

\altaffiltext{1}{Center for Cosmological Physics, The University of Chicago, 5640 South Ellis Avenue Chicago, IL  60637}
\altaffiltext{2}{Department of Astronomy and Astrophysics, The University of Chicago, 5640 South Ellis Avenue, Chicago, IL 60637.}
\altaffiltext{3}{Department of Physics and Astronomy, University of Pittsburgh, 3941 O'Hara Street, Pittsburgh, PA 15260.}
\altaffiltext{4}{Fermi National Accelerator Laboratory, P.O. Box 500, Batavia, IL 60510.}
\altaffiltext{5}{Department of Physics, University of Michigan, 500 East University, Ann Arbor, MI 48109-1120.}

\begin{abstract}
We present weak lensing measurements of the galaxy-mass
cross-correlation function over scales 0.02 to 10 $h^{-1}$ Mpc in the Sloan Digital
Sky Survey. Using a flux-limited sample of \numspec\ lens galaxies with 
spectroscopic redshifts and $\langle L\rangle \sim L_*$ and 
\numsource\ source galaxies with photometric redshifts, we find 
the mean lensing signal $\Delta \Sigma$ 
is consistent with a power-law 
galaxy-mass correlation function, $\xi_{gm} = (r/r_0)^{-\gamma}$, 
with best-fit parameters $\gamma
= 1.76 \pm 0.05$ and $r_0 = 6.1 \pm 0.7$.  The slope of $\xi_{gm}$ is
consistent with that of the galaxy auto-correlation function 
measured for a similarly selected sample of SDSS galaxies, 
$\gamma_{gg} = 1.80 \pm 0.02$, implying that the
bias parameter, $\langle b_{gm}\rangle 
\equiv \xi_{gg}/\xi_{gm} = 0.9 \pm 0.2$, 
is approximately scale independent 
over the range $r \sim 0.1 - 10 ~ h^{-1}$ Mpc, within the 
uncertainties.  We split the lens galaxy 
sample into subsets based on luminosity, color, spectral
type, and velocity dispersion, and see clear trends of the lensing 
signal with each of these parameters. The amplitude and logarithmic 
slope of $\xi_{gm}$ increases with galaxy luminosity; this trend with 
luminosity also appears in the subsample of red galaxies, which are 
more strongly clustered than blue galaxies. 
\end{abstract}

\keywords{cosmology:observations --- dark matter --- gravitational lensing ---
large-scale structure of the universe}


%\epsscale{0.8}

%\tableofcontents

\section{Introduction} \label{intro}



The measurement of galaxy clustering has long been a primary tool in 
constraining structure formation models and cosmology. Yet the power of 
galaxy surveys to discriminate between models is partially  
compromised by the fact that they provide only an indirect
measure of the underlying mass distribution, subject to considerable uncertainties 
in the {\it bias}, that is, in how luminous 
galaxies trace the mass. In this context, the galaxy-mass cross-correlation 
function $\xi_{gm}$ 
can provide important additional information, since it is in some   
sense a `step closer' to the clustering of mass. Moreover, comparing 
$\xi_{gm}$ with the galaxy auto-correlation function $\xi_{gg}$ 
yields a measure of the bias and therefore a constraint on theories 
of galaxy formation. In this paper we 
measure the correlation between galaxies and mass using weak gravitational
lensing.

In the current paradigm of structure formation, the formation of galaxies 
is heuristically divided into two parts. On large scales, cosmological parameters and the properties of the dark matter determine 
the growth of density perturbations and the eventual formation of 
massive dark halos. On smaller scales, hydrodynamic and other processes 
shape how luminous galaxies form within dark matter halos and how they 
evolve as halos accrete and merge. A natural consequence of this picture 
is that the galaxy distribution is related to but differs in detail from the mass 
distribution. This difference arises in part 
because halos are more strongly clustered than the 
dark matter as a whole, and more massive halos are more strongly clustered
than less massive ones \citep{Kaiser84}. In addition, the efficiency of 
forming luminous galaxies of different types and luminosities 
varies (primarily) with halo mass. 

The galaxy 
bias parameter can be defined in a number of ways, but a traditional 
one is as the ratio of the
galaxy and mass auto-correlation functions at fixed
separation \citep{Kaiser84},
\begin{equation} \label{bias_def}
b^2 = \frac{\xi_{gg}}{\xi_{mm}} ~~.
\end{equation}
The amplitude of the galaxy-mass cross-correlation $\xi_{gm}$   
relative to $\xi_{gg}$ and $\xi_{mm}$ can be expressed  
in terms of the correlation coefficient \citep{Pen98}, 
\begin{equation} \label{r_def}
r = \frac{\xi_{gm}}{(\xi_{mm}\xi_{gg})^{1/2}} ~,
\end{equation}
where $r \in [-1,1]$, so that $\xi_{gm} = b r \xi_{mm}$. 
In general, $b$ and $r$ can be time-dependent functions of the pair separation 
and depend on galaxy properties.  
Since we will be comparing the galaxy-galaxy and galaxy-mass 
correlation functions, we will constrain 
the `galaxy-mass' 
bias parameter, which we define as 
\begin{equation} \label{bgmdef}
b_{gm} = \frac{\xi_{gg}}{\xi_{gm}} = \frac{b}{r}~.
\end{equation}

Galaxy surveys have provided a wealth of information on the 
behavior of the galaxy bias $b$ as a function of galaxy luminosity 
and type. For example, on scales $r < 20 h^{-1}$ Mpc, the clustering 
amplitude \xigg\ increases with luminosity 
\citep{Norberg01,Zehavi02,Norberg02,Zehavi04}, 
while the amplitude and shape of $\xi_{gg}$ 
vary systematically from early to late galaxy types 
\citep{Davis76,Norberg01,Zehavi02}. Moreover, 
the nearly power-law behavior of \xigg\, as well 
as the small departures therefrom \citep{Zehavi03}, combined with 
the assumption that the dark matter distribution is 
described by cold dark matter models, indicate that the bias is 
scale-dependent on these scales.

On larger scales, $r > 20 h^{-1}$ Mpc, there is 
evidence from higher-order galaxy correlations 
\citep{Frieman99,Szapudi02,Verde02}, 
from the cosmic shear weak lensing power spectrum 
\citep{Hoekstra02c,Jarvis03},  
and from comparison of the 2dF galaxy power spectrum \citep{Percival01} 
with the WMAP cosmic microwave background (CMB) 
temperature angular power spectrum \citep{Spergel03} 
that the 
linear galaxy bias parameter $b_{lin}$ is of order unity for optically selected  
$L_*$ galaxies. 
(Here, $b_{lin}^2$ is the ratio of the galaxy correlation 
function to the mass correlation function computed in linear perturbation theory; 
it is related but not identical to the bias defined in Equation~\ref{bias_def}.)
Measurement of the parameter $\beta = \Omega_m^{0.6}/b_{lin}$ from 
redshift space distortions in galaxy surveys, combined with independent 
evidence that $\Omega_m \simeq 0.3$, also indicates $b_{lin}(L_*) \simeq 1$ 
on large scales \citep{PeacockNature2001}.

In recent years, weak gravitational lensing has become a powerful tool 
for probing the distribution and clustering of mass in the Universe. We focus on 
galaxy-galaxy lensing, the distortion induced in the images of background (source) 
galaxies by foreground lens galaxies. Although the typical distortion induced by a 
galaxy lens is tiny ($\sim 10^{-3}$) 
compared to the intrinsic ellipticities of the source 
galaxies ($\sim 0.3$), the signals from a large sample of lens galaxies can be 
stacked, providing a mean measurement with high signal to noise.  
The mean lensing signal can be used to infer the galaxy-mass cross-correlation 
function which, when compared with 
the galaxy auto-correlation function, constrains the amplitude and 
scale dependence of the bias. 

The first detection of galaxy-galaxy lensing was made by \citet{Brainerd96},  
and the field has progressed rapidly since then 
\citep{Dell96,Griffiths96,Hudson98,fis00,Wilson01,Smith01,Mckay02,Hoekstra03a}.
The first high S/N measurements were made in the Sloan Digital Sky Survey
(SDSS) \citep{fis00}.  Recent studies have benefitted from improved data 
analysis and reduction techniques 
and from surveys which are specifically designed for lensing \citep{Hoekstra01b}.
Most work in galaxy lensing has concentrated on its power to constrain galaxy
halo parameters \citep{Brainerd96,Hudson98,fis00,Hoekstra03a}.  However, galaxy 
lensing has also been
used to measure the bias directly \citep{Hoekstra01b,Hoekstra02b}; 
their results indicate that $b$ and $r$ are scale-dependent over 
scales $\sim 0.1-5 h^{-1}$ Mpc, but that the ratio $b/r$ is nearly 
constant at $b/r \simeq 1.1$ over this range. 

A significant step forward in galaxy-galaxy lensing came with the use of samples of 
lens galaxies with spectroscopic 
redshifts \citep{Smith01,Mckay02}.  Lensing measurements could then be 
made as a function of physical rather than angular separation, placing
lensing correlation measurements on a par with the auto-correlation measurements
from galaxy redshift surveys. Incorporation of photometric redshifts 
for the source galaxies \citep{Hudson98} also substantially reduces errors 
in the lens mass calibration due to  
the breadth of the source galaxy redshift distribution. 


In this paper, we study galaxy-mass correlations in the SDSS using weak
gravitational lensing.  Using a sample of \numspec\ galaxies with 
spectroscopic redshifts and \numsource\ galaxies with photometric redshifts, 
we measure the lensing signal with high S/N over scales from $0.02-10 h^{-1}$ Mpc.  
This is the first galaxy lensing study to incorporate both spectroscopic 
lens redshifts and photometric source redshifts, it is by far the 
largest galaxy lens-source sample compiled, and it extends to scales 
larger than previous galaxy-galaxy measurements. A similar spectroscopic 
sample has been used for galaxy auto-correlation 
measurements in the SDSS \citep{Zehavi03}. 
We compare the galaxy-mass and galaxy-galaxy correlations to constrain $b_{gm}$ 
over scales from 100 kpc to 10 Mpc.  We also use the spectroscopic
and photometric data from the SDSS to divide the lens galaxy sample by 
luminosity, color, spectral type, and velocity dispersion.  We see clear
dependences of $\xi_{gm}$ on each of these properties.

The layout of the paper is as follows: In \S \ref{lensing} we introduce lensing
and the measurement methods.  In \S \ref{data} we discuss the SDSS data, reductions,
and sample selection.  The basic measurement of the lensing 
signal \deltasig\ is presented in \S
\ref{gglmeas:deltasig}, and important checks and corrections for systematic
errors using random points are discussed in \S \ref{gglmeas:random}. In \S
\ref{gglmeas:wgm} we use the data to infer the galaxy-mass correlation 
function $\xi_{gm}$ and compare it with independent measurements 
of $\xi_{gg}$ to constrain the bias.
In \S \ref{gmcflum}-\ref{gmcftype} we explore 
the dependence of galaxy-mass correlations on galaxy luminosity and type.  
We conclude in \S \ref{discussion} and discuss 
possible systematic errors in the Appendix.

Throughout this paper, 
where necessary we use a Friedman-Robertson-Walker cosmology with
$\Omega_M$ = 0.27, $\Omega_{\Lambda}$ = 0.73, and H$_0$ = 100 h km/s.

\section{Lensing and Galaxy-mass Correlations} \label{lensing}

\subsection{Gravitational Shear and the Galaxy-mass correlation function} 
\label{lensing:shear}

In this section 
we review the relation between the induced 
shear, which can be estimated from source galaxy shape measurements, 
the 
galaxy-mass cross correlation function $\xi_{gm}$, and the projected 
cross-correlation function \wgm. 
The tangential shear, $\gamma_T$, azimuthally averaged over a thin annulus 
at projected radius $R$ from a lens galaxy, is
directly related to the projected surface mass density of the lens within the
aperture, 
\begin{equation} \label{eq:gammat}
\gamma_T \times \Sigma_{crit} =  \overline{\Sigma}(< R) -\overline{\Sigma}(R)
\equiv \Delta \Sigma ~,
\end{equation}
where $\overline{\Sigma}(< R)$ is the mean surface density within radius $R$, 
and $\overline{\Sigma}(R)$ is the azimuthally averaged surface density at
radius $R$ \citep{Escude91,Kaiser94,Wilson01}.  The proportionality constant
$\Sigma_{crit}$ encodes the geometry of the lens-source system,
\begin{equation} \label{eq:sigmacrit}
\Sigma_{crit} = \frac{4 \pi G D_{LS} D_L}{c^2 D_S}~,
\end{equation}
where $D_{L}$, $D_S$, and $D_{LS}$ 
are angular diameter distances to lens, source, and between
lens and source.  

Due to the
subtraction in equation \ref{eq:gammat}, uniform mass sheets (such as the mean density
of the universe $\overline{\rho}$) 
do not contribute to \deltasig---it measures the mean {\it excess}
projected mass density.
The mean excess mass density at radius $r$ from a galaxy is
$\overline{\rho}~\xi_{gm}(r)$. The mean excess projected 
density $\Sigma(R)$ is given by the radial integral:
\begin{equation}
\langle \Sigma(R) \rangle = 
      \int \overline{\rho} \xi_{gm}(x,y,z) dz \equiv \overline{\rho} w_{gm}(R) ~, 
\end{equation}
where \wgm\ is the projected galaxy-mass correlation function and $R = (x^2 +
y^2)^{1/2}$ is the projected radius.  The observable \deltasig\ is itself an integral
over $\Sigma(R)$ and hence \wgm:
\begin{equation} \label{deltasig_and_wgm}
\langle \Delta\Sigma(R) \rangle = \overline{\rho} \times \left[ \frac{2}{R^2} 
\int_0^R R^\prime dR^\prime ~w_{gm}(R^\prime) -w_{gm}(R) \right]
\end{equation}

If the cross-correlation function can be approximated by a power-law in separation, 
$\xi_{gm} = (r/r_0)^{-\gamma}$, then \wgm\ can
be written as
\begin{equation} \label{xigm_def}
w_{gm}(R)  =  F(\gamma, r_0) R^{1-\gamma} ~,
\end{equation}
where $ F(\gamma, r_0) = r_0^{\gamma}
\Gamma(0.5)\Gamma[0.5(\gamma-1)]/\Gamma(0.5\gamma)$ \citep{DavisPeebles83}.  In
that case, the mean lensing signal \deltasig\ is also a power law with index $\gamma
- 1$ and is simply proportional to $\overline{\rho}w_{gm}$,
\begin{equation} \label{deltasig2wgm}
\langle \Delta \Sigma (R)\rangle = \left(
\frac{\gamma -1}{3-\gamma} \right)\overline{\rho} w_{gm}(R)~.
\end{equation}

More generally, 
the galaxy-mass cross-correlation function is given by 
the inversion formula \citep{Saunders92}, 
\begin{eqnarray} \label{Abel}
\xi_{gm}(r) &=&  -\frac{1}{\pi} \int_r^\infty dR \frac{dw_{gm}/dR}{(R^2 - r^2)^{1/2}}
\nonumber \\
&=& \frac{1}{\overline{\rho}\pi} \int_r^\infty dR \frac{\left({d\Delta\Sigma \over
dR}+{2\Delta\Sigma \over R}\right)}{(R^2 - r^2)^{1/2}} ~.
\label{eq:invert}
\end{eqnarray}
Assuming $\Delta \Sigma(R)$ is measured at a set of discrete 
values $R_j < R_{max}$, the correlation function 
can be estimated by interpolation,
\begin{eqnarray}
\xi_{gm}(r=R_i) &=& {1\over \overline{\rho} \pi}   \sum_{j \geq i} \left({\Delta \Sigma_{j+1} -
\Delta \Sigma_j \over R_{j+1}-R_j} + {\Delta \Sigma_j \over R_j} 
+ {\Delta \Sigma_{j+1} \over R_{j+1}}\right) \nonumber \\
& \times & \ln \left({R_{j+1}+ \sqrt{R^2_{j+1}-R^2_i} \over 
R_j + \sqrt{R^2_j - R^2_i}}\right) + \xi_{gm}(R_{max}) \nonumber \\
\label{eq:interp}
\end{eqnarray}
where the last term reminds us of the (unknown) 
contribution from scales beyond those for which we have measurements;  
it can be thought of as another manifestation of the lensing mass 
sheet degeneracy.
Provided $\xi_{gm}$ falls sufficiently fast with separation, this term 
is negligible for scales $r$ smaller than a fraction of $R_{max}$. 
Furthermore, since $\xi_{gm}$ is linear in $\Delta \Sigma$, 
the covariance matrix of the latter can be straightforwardly 
propagated to that of the former.


\subsection{Estimating \deltasig}\label{lensing:deltasig}

We estimate the shear by measuring the tangential component of the source galaxy 
ellipticity relative to the the lens center, $e_+$, 
also known as the E-mode.  In general, the shear is
related in a complex way to $e_+$ \citep{SchneiderSeitz95}, but in the weak lensing 
regime the relationship is linear:
\begin{equation} \label{eq:ellip_induce}
e_+ = 2 \gamma_T \mathcal{R} + e_+^{int} ~,
\end{equation} 
where $e_+^{int}$ is the intrinsic ellipticity of the source, $\gamma_T$ is the
shear, and $\mathcal{R}$ is the ``responsivity'' (see equation 
\ref{eq:responsivity}).  The assumption behind weak lensing measurements is
that the source galaxies are randomly oriented in the absence of lensing, in which 
case their intrinsic shapes constitute a large but random source of error on the shear
measurement.  
This ``shape noise'' is the
dominant source of noise for most weak lensing measurements. 

***Presumably we should mention intrinsic alignments here, with 
some references and soothing remarks about why it's not an issue.***

The other component of the ellipticity, $e_\times$, also known as the B-mode, 
is measured at 
45\degr\ with respect to the tangent. The average B-mode should be zero if the 
induced shear is due only to gravitational 
lensing \citep{Kaiser95,Luppino97}.  This provides an important test for 
systematic errors, such as PSF smearing, since they generally  
contribute to both the E- and B-modes.


In order to estimate \deltasig\ from the shear, we must know the angular 
diameter distances $D_L$, $D_S$, $D_{LS}$ for 
each lens-source pair (see equations \ref{eq:gammat} and \ref{eq:sigmacrit}). 
In the SDSS, we have spectroscopic redshifts for all the lens galaxies, so 
that $D_L$ is measured to high precision (assuming a cosmological model). 
For the source galaxies, we have photometric redshift estimates (photoz), 
with typical relative errors of  
20-30\% (see \S \ref{data:imaging:photoz}), so there is some uncertainty in 
the value of $\Sigma_{crit}$ for each lens-source pair. 

Given the known redshifts of the lenses, the distribution of
errors in the source galaxy 
ellipticity, and the distribution of errors in the photometric
redshift for each source, we can write the likelihood for $\Delta \Sigma$ from 
all lens-source pairs,
\begin{equation} \label{eq:deltasig_like}
\mathcal{L}(\Delta\Sigma) = \prod_{j=1}^{N_{Lens}} \prod_{i=1}^{N_{Source}}
\int dz_s^i P(z_s^i) P(\gamma_T^i | z_s^i, z_L^j) ~,
\end{equation}
where $\gamma_T^i = e_+^i/2 \mathcal{R}$ is the shear estimator for the $i$th source
galaxy, $P(z_s^i)$ is the probability distribution for its redshift 
(the product of the Gaussian error  
distribution returned by the photoz estimator and a prior based on the 
redshift distribution for the source population; 
see \S \ref{data:imaging:photoz}),
and $P(\gamma_T^i | z_s^i, z_L^j)$ is the probability distribution of the shear
given the source and lens redshifts, which is a function of the desired
quantity \deltasig:
\begin{eqnarray} \label{eq:pgamma}
\lefteqn{P(\gamma_T^i | z_s^i, z_L^j) \varpropto}   \nonumber \\
& &
\exp
\left(-\frac{1}{2} 
\left[
\frac{\gamma_T^i - \Delta\Sigma \times \Sigma_{crit}^{-1} (z_s^i, z_L^j)}{\sigma(\gamma_T^i)}.
\right]^2
\right) 
\end{eqnarray}
In equation \ref{eq:pgamma}, $\sigma^2(\gamma_T^i) = (\sigma^2(e_+^i) +
\sigma^2_{SN})/4\mathcal{R}^2$:  
the shear uncertainty is the sum of the measurement variance 
$\sigma^2(e_+^i)$ and the intrinsic variance in the shapes of the source
galaxies $\sigma^2_{SN}=\langle (e_+^{int})^2\rangle$. The shape noise 
measured from bright, well-resolved galaxies is $\sigma_{SN} \approx 0.32$, and
the typical measurement error $\sigma(e_+)$ ranges from $\sim$0.05 for \rmag=18 to
$\sim$0.4 for \rmag=21.5.  The intrinsic shape distribution is not Gaussian as we have
assumed in equation \ref{eq:pgamma}, but it is symmetric.  Monte Carlo
simulations indicate that this approximation does not bias the measurement 
of $\Delta \Sigma$ within our
measurement uncertainties, provided we take $\sigma_{SN}$ as the standard deviation
of the non-Gaussian shape distribution.

Although the typical uncertainty in the source galaxy 
photometric redshifts is 20-30\%, this
is small compared to the relative shear noise, which is typically 
$\sigma_{SN}/\gamma_T \sim 300$\%.
Assuming shape error is the dominant source of noise, we can approximate equation
\ref{eq:deltasig_like} as
\begin{equation}
\log \mathcal{L}(\Delta\Sigma) = 
%\sum_{j=1}^{N_{Lens}} \sum_{i=1}^{N_{Source}}
\sum_{j,i}
\left(-\frac{1}{2} 
\left[
\frac{\gamma_T^i - \Delta\Sigma \times \langle \Sigma_{crit}^{-1}\rangle_{j,i}}{\sigma(\gamma_T^i)}
\right]^2
\right)
\end{equation}
where we now use the critical surface density averaged over the photoz
distribution for each source galaxy, 
\begin{equation} \label{eq:meansigcrit}
\langle \Sigma_{crit}^{-1} \rangle_{j,i} = \int dz_s^i P(z_s^i) \Sigma_{crit}^{-1}(z_s^i, z_L^j)
\end{equation}
Monte
Carlo simulations indicate that the true likelihood approaches this Gaussian
approximation after stacking only a few hundred lenses.

The maximum likelihood solution is the standard weighted average,
\begin{equation}
\Delta\Sigma =  
{\sum_{j=1}^{N_{Lens}} \sum_{i=1}^{N_{Source}} \Delta\Sigma_{j,i} w_{ji}
\over \sum_{j=1}^{N_{Lens}} \sum_{i=1}^{N_{Source}}w_{ji}} ~,
\end{equation}
where  
\begin{eqnarray}
\Delta\Sigma_{j,i} & = & \gamma_T^i/\langle \Sigma_{crit}^{-1}\rangle_{j,i} \\
\sigma_{j,i} & = & \sigma(\gamma_T^i)/\langle \Sigma_{crit}^{-1}\rangle_{j,i}
\end{eqnarray}
and the weights $w_{ji} = 1/\sigma_{j,i}^2$. 
Although this simple inverse variance weighting
is not optimal \citep{Bern02}, it does lead to unbiased results and 
does not overestimate the errors significantly.


As indicated by equation \ref{eq:ellip_induce}, the ellipticity induced by a shear
depends on the object's shear responsivity $\mathcal{R}$. 
A measure of how an applied shear alters the shape of a source, $\mathcal{R}$ 
depends on the object's intrinsic ellipticity and is similar to
the shear polarizability of \citet{ksb95}. Following \citet{Bern02},  
we calculate a mean
quantity for the responsivity as a 
weighted average over the tangential ellipticities.
\begin{equation}
\mathcal{R} = \frac{ \sum_i w_i \left[ 1 - k_0 - k_1 e_+^2\right] }{\sum_i w_i}~,
\label{eq:responsivity}
\end{equation}
where we have again assumed a Gaussian distribution of ellipticities so 
that $k_0$ and $k_1$ are simple, 
\begin{eqnarray}
k_0 & = & (1-f) \sigma^2_{SN},~~~k_1 = f^2, \nonumber   \\
f & = & \frac{\sigma^2_{SN}}{\sigma^2_{SN} + \sigma^2(e_+^i)}
\end{eqnarray}
and the weights are given by
\begin{equation}
w_i = ????
\end{equation}

***This needs more explanation: presumably you calculate $\mathcal{R}$ for 
a number of bins in $e_+$? How are the weights $w_i$ defined here? I assume 
it's just $w_i = 1/\sigma^2(\gamma_T^i)$? 
Also, in your version, the last expression above was $\sigma^2_e$: have 
I replaced it with the correct expression? ***

\section{Data} \label{data}

The Sloan Digital Sky Survey (SDSS; \cite{York00}) is an ongoing project to map
nearly 1/4 of the sky in the northern Galactic cap (centered at 12\hrs
20\minutes, +32.8\degr).  Using a dedicated 2.5 meter telescope
located at Apache Point Observatory in New Mexico, the SDSS comprises a photometric
survey in 5 bandpasses (\ugriz; \citet{Fukugita96}) to \rmag\ $\sim$ 22 
and a spectroscopic survey of galaxies, luminous red galaxies, quasars, 
stars, and other selected targets. In
addition, the survey covers 3 long, thin stripes in the southern Galactic 
hemisphere; the central southern stripe, covering 
$\sim 200$ square degrees, will be imaged many times, 
allowing time-domain studies as well as a deeper co-added image.

To date, the SDSS has released 2099 square degrees of imaging data to the
public: the SDSS Early Data Release (EDR; \citet{Stough02}) and the First Data
Release (DR1; \citet{Abaz03}).  DR1 includes about 186,000 spectra of
galaxies, quasars, and stars.  In this paper, we make use of a larger data set 
comprising nearly 3800 square degrees of imaging data and \numspectot\ galaxies with
spectra. 

***Is the number above for main galaxies only? or all spectra classified as 
galaxies?***

We select our lens galaxies from the SDSS main galaxy spectroscopic sample.  
For our background 
sources we use only well-resolved galaxies drawn from the photometric survey 
with well-measured photometric redshifts. Each sample is described in detail below.

\subsection{Imaging Data} \label{data:imaging}

Imaging data are acquired in time-delay-and-integrate (TDI) or drift scan mode.
An object passes across the camera at the same rate the CCDs read out, which
occurs continuously during the exposure.  The object crosses each of the 5
SDSS filters in turn, resulting in nearly simultaneous images in each bandpass.
In order for the object to pass directly down the CCD columns, the distortion
across the field of view must be exceedingly small.  This is advantageous for
lensing, since distortions in the optics cause a bias in galaxy shapes (\S
\ref{data:imaging:correct}).
The distortion in the optics of the SDSS 2.5m telescope is negligible 
\citep{Stough02} and can be ignored. 

The imaging data are reduced through various software pipelines, including 
the photometric (\photo,
\cite{Lupton10}), astrometric \citep{Pier03}, and calibration \citep{Smith02} pipelines, 
leading to calibrated lists of detected objects.
The calibrated object lists are subsequently fed through various target
selection pipelines \citep{Eisenstein01,Strauss02,Richards02} which select 
objects for spectroscopic followup.


The shape measurements discussed in \S \ref{data:imaging:meas} are implemented in
\photo\ (\texttt{v5\_3}), so we work directly with the calibrated object lists.
These lists contain, among many other things, position (RA,DEC), several 
measures of the 
flux, diagnostic flags for the
processing, and moments of the light distribution for each object and for the 
local Point Spread Function (PSF) \citep{Stough02}.
We augment the parameters measured by \photo\ with the probability that each
object is a galaxy (\S \ref{data:imaging:sg}) and with photometric redshifts.

\subsubsection{Shape Measurements} \label{data:imaging:meas}

Weak lensing measurements rely on the assumption that source galaxy shapes are an
unbiased, albeit rather noisy, measure of the shear induced by foreground
lenses.  Therefore high S/N shape measurements and accurate corrections
for bias are crucial for weak lensing measurements.  For both shape
measurements and corrections we use techniques described in \cite{Bern02} 
(hereafter BJ02).

We determine the apparent shapes of objects from their flux-weighted second
moments.
\begin{equation}
Q_{m,n} = \sum_{m,n} I_{m,n} W_{m,n} x_m x_n ,
\label{data:imaging:meas:mom}
\end{equation} 
where $I_{m,n}$ is the intensity at pixel $m,n$ and $W_{m,n}$ is 
an elliptical Gaussian weight function, iteratively adapted to
the shape and size of the object.  Initial guesses for the size and position
of the object are taken from the Petrosian radius and \photo\ centroid. 
Objects are removed if the iteration does not converge or if the iterated centroid
wanders too far from the \photo\ centroid.
The shape is parametrized by the polarization, or ellipticity, components, 
defined in terms of the second moments, 
\begin{eqnarray}
e_1 & = & { {Q_{11} - Q_{22}} \over {Q_{11} + Q_{22}} }  \\
e_2 & = & { {2Q_{12}} \over {Q_{11} + Q_{22}} }.
\end{eqnarray} \label{data:imaging:meas:e1e2}
The \polarization\ is related directly to the shear via equation \ref{eq:ellip_induce}.

Because the natural coordinates for the SDSS are survey coordinates
$(\lambda,\eta)$ \citep{Stough02},
we rotate the ellipticities into that coordinate system for the lensing
measurements.  The full covariance matrix for $(e_1,e_2)$ is used to transform
the errors under rotations in the standard way.  We combine the shape
measurements from the \gmag, \rmag, and \imag\ bandpasses using the covariance
matrices.  This increases the S/N, simplifies the analysis, and reduces
bandpass-dependent systematic effects.  The \umag\ and \zmag\ bands are much
less sensitive and would contribute little to the analysis.

***Alittle more explanation needed here: (1) a sentence or two on how the 
shape measurement errors are estimated; (2) alittle more on how the covariance matrices 
are measured--i.e., using means as fn. of magnitude or size or.... ***

\subsubsection{PSF Reconstruction} \label{data:imaging:psf}

An anisotropic PSF, caused by instrumental and atmospheric effects, 
smears and alters the shapes of galaxies in a way which can mimic
lensing. In addition, the finite size of the PSF---the seeing---tends to
circularize the image, reducing the measured ellipticity.
In order to correct for these effects, one should, in
principle, determine the exact shape and size of the PSF at the position of 
every source galaxy.  Imaging in drift scan mode produces 
images that are long, thin stripes
on the sky.  Because the PSF varies over time during a scan  (which 
can last for several hours), it must be tracked as a function
of position in the image (see \S
\ref{data:imaging:correct}).

The photometric pipeline uses Karhunen-Lo\`{e}ve (KL) decomposition
\citep{Hot33,Kar47,Loeve48} to model the PSF. The PSF is modeled on a
frame-by-frame basis, where frames are defined as $2048\times 1490$
pixel chunks composing the long SDSS image.  
A set of bright, isolated stars are chosen from 
$\pm 2$ frames around the central frame, yielding typically $15-25$ stars.  
These stellar images $P_{(i)}(u,v)$ are used to form a set of KL basis functions or 
\emph{eigenimages} $B_r(u,v)$, in terms of which the images can be reconstructed 
by keeping the first $n$ terms in the expansion, 
\begin{equation}
P_{(i)}(u,v) = \sum_{r=0}^{n-1}a_{(i)}^r B_r(u,v)
\end{equation}
where $P_{(i)}$ denotes the $i^{th}$ star, and $u,v$ are 
pixel positions relative to the object center.

The spatial dependence of the coefficients $a_{(i)}^r$ are determined
via a polynomial fit, 
\begin{equation}
a_{(i)}^r \approx \sum_{l=m=0}^{l+m \le N} b_{lm}^r x_{(i)}^l y_{(i)}^m
\end{equation}
where $x,y$ are the coordinates of the center of the $i^{th}$ star 
relative to the center of the frame, $N$
is the highest order in $x,y$ included in the expansion, and $b_{lm}^r$ are
determined from minimizing 
\begin{equation}
\chi^2 = \sum_i\left( P_{(i)}(u,v) - \sum_{r=0}^{n-1} a_{(i)}^r B_r(u,v)\right)
\end{equation}
Only the stars on $\pm 1/2$ a frame surrounding the given frame are 
used to determine the spatial variation.

In \photo, the number of terms used from the KL basis is usually $n=3$; 
the order of the spatial fit is $N=2$ unless there are too few
stars, in which case the fit may be order 1 or even 0 (rare). To determine 
the coefficients $b_{lm}^r$, 
a total of $n(N+1)(N+2)/2$ constraints are needed, which may seem like too
many for the typical $15-25$ stars.
There are many pixels in each star, however,
so the number of spatial terms $(N+1)(N+2)/2 = 6$ (for quadratic fits) 
should be compared with the number of available PSF estimators.

***But you don't use all 5 frames for the spatial variation, only ~2 frames, 
which means ~6-10 stars, no? Or is it in fact 15-25 stars per frame?***

With the spatial fit above, 
the PSF is reconstructed at the position of each object and its second moments 
are measured; these are used in the analytic
PSF correction scheme described in \S \ref{data:imaging:correct}.

\subsubsection{Shape Corrections} \label{data:imaging:correct}

To correct galaxy shapes for the effects of
PSF dilution and anisotropy, we use the techniques of BJ02 
with the modifications specified in
\cite{Hirata02}.  Rather than a true deconvolution, this is an approximate
analytic technique. The PSF is modeled as a transformation of the preseeing
shape; to remove the effects of the PSF, the inverse transform must
be calculated.  This transformation is performed in shear space (or
\polarization\ space):
\begin{equation}
R * [ (-e_{PSF}) \oplus e ] = (-e_{PSF}) \oplus e_0
\end{equation}
where $e$ is the preseeing \polarization, $e_0$ is the measured \polarization,
$e_{PSF}$ is the \polarization\ of the PSF, the resolution
parameter is $R=1-P$, and $P$ is the smear polarizability.  The
operator $\oplus$ is the shear addition operator defined in BJ02.

For unweighted moments ($W_{m,n}=1$) or if the objects and PSF 
have Gaussian surface brightness profiles, the formulae in BJ02 are
exact, and the resolution parameter in this case is just 
$R=1-(r_{PSF}/r_{obj})^2$, where $r$ is the linear size of the object or PSF.
The light profiles of galaxies and of the PSF differ 
significantly from Gaussian, however.  Furthermore, as noted in 
\S \ref{data:imaging:meas}, we use a Gaussian radial weight
function to optimize the S/N of object shape measurements.  
As a consequence, the resolution
parameter $R$ must be derived in an approximate way that accounts for both the
weight function and the non-Gaussianity of the light profiles.  
For this work we use a weighted fourth 
moment to correct for higher order effects.

***I agree with Dave: you definitely need to make this section less opaque. 
Is the equation I added 
above correct? A short paragraph describing how you get $R$ would be good. Where you 
mention smear polarizability above is it appropriate to reference 
Kaiser etal 95? Also, some words on how well the shape corrections do as 
a function of object or shape S/N would be useful if you have that info.***


\subsubsection{Star-Galaxy Separation} \label{data:imaging:sg}

To separate stars and galaxies cleanly at all magnitudes, we use the Bayesian
method discussed in \cite{Scranton02}.  The method makes use of the
concentration parameter, which can be calculated from parameters output by
\photo.  The concentration is the difference between the object's PSF magnitude
and exponential disk magnitude. The PSF magnitude is derived by fitting the
local PSF shape to the object's light profile. The only free parameter in this
fit is the overall flux. An exponential disk is also fit to the object, but in
that case the scale length is also a free parameter.  Thus, large objects 
have more flux in the exponential than the PSF fit and correspondingly large 
concentration, while stars have concentration around zero.

At bright magnitudes, galaxies and stars separate cleanly in concentration
space. At fainter magnitudes, photometric
errors increase and the distributions overlap. This is demonstrated in
figure \ref{fig:conc}, which shows the distribution of concentration for
objects with $20 <~$\rmag$~< 21$ and $21 <~$\rmag$~< 22$ 
drawn from 100 fields (***is a field the same as a frame? if not explain***)
of a single SDSS imaging run (3325), with mean seeing of 1.25\arcsec (typical 
of SDSS image quality).
In bad seeing conditions, a smaller percentage of galaxies are larger than the
PSF, again making separation difficult. 

\begin{figure}[htbp]
%\centering
%\includegraphics[width=240pt]{concentration.eps}
%\caption{
\plotone{concentration.eps} \figcaption{
Concentration distribution for objects
with $20 <~$\rmag$~< 21$ (dark curve) and $21 <~$\rmag$~< 22$ (light
curve). Stars have concentration near zero.  At faint magnitudes, stars and
galaxies are not as easily separated. \label{fig:conc} }
\end{figure}

If we know the
distribution in concentration for galaxies and stars as a function of seeing
and magnitude, we can assign each object a probability $P_g$ that it is a galaxy, 
given its concentration, magnitude, and the local seeing.  
For our source galaxy sample, we then 
select objects which have high values of $P_g$. 
To map out the distribution in concentration, we use regions from the SDSS 
Southern Survey which have been imaged many times.  Some of these regions have been
imaged as many as 16 times, with an average of about 8. By averaging the flux
for each object from the multiple exposures to obtain higher S/N, 
a clean separation between the star and galaxy concentration distributions 
is achieved to fainter magnitudes.   
We have maps of the concentration distribution for $16 <~$\rmag$~< 22$ and
$0.9 <~$seeing$~< 1.8$, allowing us to accurately calibrate $P_g$ 
and therefore define a clean galaxy sample.
In the few regions of very bad seeing, we extrapolate 
conservatively, erring on the side of including fewer galaxies in the sample.

***What is extrapolated?***

\subsubsection{Photometric Redshifts} \label{data:imaging:photoz}

A photometric redshift (photoz) is estimated for each object in the source catalog.
The repaired template fitting method is used, described in detail in
\cite{Csabai2000} and implemented in the SDSS EDR \citep{Csabai2003} as well as
the DR1 \citep{Abaz03}.  This technique uses the 5-band photometry for each
object as a crude spectrum.  The algorithm compares this spectrum to templates
for different galaxy types at different redshifts.  The result is an estimate
of the type and redshift of each galaxy.


There is a large covariance between the inferred type of the galaxy and its photometric 
redshift.  The code outputs a full covariance matrix for type and redshift.
Because we do not use the type information, we use the error marginalized over
type. We further assume that the resulting error is Gaussian, which is only a 
good approximation if the estimated redshift is large compared to its error.  
This introduces a 
bias in the estimate of $\langle \Sigma_{crit}^{-1} \rangle$, 
but Monte Carlo simulations indicate that this is a negligible effect. 

From comparisons to galaxies with known redshifts, the {\it rms}
in the SDSS photoz 
estimates is found to be $\sim 0.035$ for \rmag\ $< 18$, increasing to $\sim
0.1$ for \rmag\ $< 21$.  About 30\% of our source galaxy sample has \rmag\
between 21 and 22.  Although the photozs are less reliable in this magnitude range, 
these objects receive relatively little weight in the analysis, because they have
large shape errors and large photoz errors (and hence small $\langle
\Sigma^{-1}_{crit} \rangle$).


The photoz distribution for the source sample (see \S \ref{data:imaging:sample}) for 
SDSS stripes $9-15$ \citep{ProjBook}
is shown in figure \ref{fig:photozdist}.  The histogram shows the 
photometric redshifts, and the smooth curve is the distribution calculated by
summing the Gaussian distributions for each galaxy.  We use this smooth curve as a
prior on the photometric redshift when calculating the inverse critical density
for each lens-source pair (see \S \ref{lensing:deltasig}).



\begin{figure}[htbp]
\plotone{photoz_stripe09_10_11_12_13_14_15.eps} \figcaption{
Distribution of photometric redshifts for sources identified in
SDSS stripes $9-15$.  The histogram shows the photozs in bins of $\Delta z=.01$, 
and the smooth curve is derived from summing the Gaussian distributions associated
with each object.\label{fig:photozdist} }
\end{figure}

Although there is significant overlap between the distribution of photozs and
the distribution of lens redshifts (figure \ref{fig:zhist}), sources with
photozs in front of or near the lens redshift are given appropriately
small weight according to equation \ref{eq:meansigcrit}.

\subsubsection{Defining the Source Sample} \label{data:imaging:sample}

Source galaxies are drawn from SDSS imaging stripes 9-15 and 27-37.  
***Need to say something about where these regions are in RA,DEC and/or show an 
Aitoff projection*** We make a series of cuts aimed 
at ensuring that the sample is of 
high purity (free from stellar contamination) and includes only well-resolved objects 
with usable shape information. We  
first require that the
extinction-corrected \rmag-band Petrosian magnitude is less than 22.  
We next make an object size cut, requiring that the 
resolution parameter $R>0.2$. 
This removes most of the stars and unresolved galaxies from the 
sample. However, at faint magnitudes (\rmag$ > 21$), 
many stars and galaxies have similar values of $R$ 
due to measurement error, so a further cut is needed. 
We employ the Bayesian galaxy probability (\S \ref{data:imaging:sg}) and  
find that the combination $R > 0.2$,  
$P_g > 0.8$ guarantees that the source galaxy catalog is greater than 99\% pure
for \rmag\ $< 21.5$ and greater than 98\% pure for $21.5 <$\rmag$<
22$. 

Additionally, we remove about $\sim$ 8\% of the sources---those 
with photoz errors greater than
0.4, and we further exclude objects with photoz less than 0.02 or greater than 0.8
since failed measurements tend to pile up at a photoz of 0.0 or 1.0.  This
removes another 10\% of the objects. 
The final source catalog contains \numsource\ galaxies, 
corresponding to a density of about $1-2$ source galaxies per square arcminute,
depending on the local seeing.

\subsection{Spectroscopic Data} \label{data:spectro}

The lens galaxies are selected from the SDSS 
``main'' galaxy spectroscopic sample, which is 
magnitude- ($14.5 <$\rmag$< 17.77$) and
surface brightness-limited ($\mu_r < 23.5$), although these limits
varied during the commissioning phase of the survey. 
See \citet{Strauss02} for a description of ``main'' galaxy target selection.

SDSS spectroscopy is carried out using 640 optical fibers positioned in 
pre-drilled holes in a large metal plate in the focal plane. 
Targeted imaging regions are assigned spectroscopic plates by an adaptive 
tiling algorithm \citep{Blanton03}, which also assigns each object a fiber. 
The spectroscopic data  are reduced to 1-d spectra by
\spectrotwo, and the \spectroone\ pipeline outputs redshift and an 
associated confidence level, spectral 
classification (galaxy, quasar, star), line measurements, and spectral 
type for galaxies, among other parameters 
\citep{Stough02}.  In addition, the velocity dispersion is  
measured for a large fraction of the early type galaxies.

For this analysis, 
we use a subset of the available spectroscopic ``main'' galaxy sample known as LSS 
\texttt{sample12} (M. Blanton 2003, personal communication). Although we draw
from a larger sample, the mask (see \S \ref{data:spectro:sphpolymasks}) was 
produced for this subset.  This sample is also
being used for analysis of the galaxy auto-correlation function 
\citep{Zehavi04}, while a slightly earlier 
sample (\texttt{sample11}) has been used to estimate the galaxy power spectrum
\citep{Tegmark04}.  Using this sample allows us to make meaningful comparisons
between the auto-correlation function and the galaxy-mass cross correlation
function.  The spectroscopic 
reductions used here are those of the 
\spectroone\ pipeline, for which redshifts and spectroscopic classifications
differ negligibly from those in the above references.

\subsubsection{Redshifts} \label{data:spectro:z}

The SDSS spectra cover the wavelength range 3800-9200\AA\ with a resolving
power of 1800 \citep{Stough02}. Repeated 15-minute exposures (totalling 
at least 45 minutes) are taken until 
the cumulative median (S/N)$^2$ per pixel 
in a fiber aperture is greater than 15 at $g=20.2$ and $i=19.9$ in all 4 
spectrograph cameras. Redshifts
are extracted with a success rate greater than 99\%, and redshift confidence
levels are greater than 98\% for 95\% of the galaxies. Repeat exposures 
of a number of spectroscopic plates indicates that ``main'' galaxy 
redshifts are reproducible to 30 km/s.  We apply a cut on the redshift confidence 
level at $>75$\%, which removes 0.5\% of the galaxies.
Many galaxies are further removed during the lensing
analysis, as discussed in \S \ref{data:spectro:sphpolymasks} and \S
\ref{data:spectro:pixelmasks}.
A redshift histogram is shown in figure \ref{fig:zhist} for the remaining \numspec\
galaxies used in this work.

***The median lens galaxy redshift is xxx.***

\begin{figure}[htbp]
%\centering
%\includegraphics[width=240pt]{redshift_hist.eps}
%\caption{
\plotone{redshift_hist.eps} \figcaption{
Redshift distribution for galaxies used
in this study.  This sample contains only ``main'' galaxy
targets.\label{fig:zhist}}
\end{figure}

\subsubsection{K-corrections} \label{data:spectro:kcorr}

We apply K-corrections using the method discussed in \citet{Blanton02}
(\texttt{kcorrect v1\_10}).  Linear combinations of four spectral templates are
fit to the five SDSS magnitudes for each galaxy given its redshift.  Rest-frame
absolute magnitudes and colors are then calculated.  Figure
\ref{fig:absmag_hist} shows the distribution of absolute Petrosian magnitude 
for each of the 5 SDSS bandpasses.

***What value of h is assumed in this Figure? Probably should specify that 
at the end of \S 1.***



\begin{figure}[tp]
\plotone{absmag_hist.eps} \figcaption{ Distribution of rest-frame absolute
magnitude in each of the 5 SDSS bandpasses for lenses used in this study.  
\label{fig:absmag_hist}}
\end{figure}


\subsubsection{Spectroscopic Masks} \label{data:spectro:sphpolymasks}

Because SDSS spectroscopy is taken through circular plates with a finite 
number of fibers of finite angular size, the spectroscopic completeness varies 
across the survey area. The resulting spectroscopic mask 
is represented by a
combination of disks and spherical polygons \citep{Tegmark04}.  Our spherical
polygon mask contains 3844 polygons covering an area of 2818 square degrees.
Each polygon also contains the completeness in that region, a number
between 0 and 1 based on the percentage of targeted galaxies in that
region which were observed.  We apply this mask and include only regions 
where the completeness is 
greater than 90\%.  These same criteria are used to generate the random points
as discussed in \ref{gglmeas:random}.

\subsubsection{Photometric Masks} \label{data:spectro:pixelmasks}

Although the shape correction for anisotropic PSF (\S \ref{data:imaging:correct}) 
substantially reduces the galaxy shape bias, it does not completely eliminate 
it---generally, a small residual, slowly varying PSF shear remains.  
Fortunately, a residual PSF bias that is constant over the lens 
aperture cancels on average from the azimuthally averaged tangential 
shear, since two aligned sources separated by 90\degr relative to the lens 
contribute with equal and opposite sign. To take advantage of this
cancellation, we divide the source galaxies around each lens into 
quadrants and demand that at least two
adjacent quadrants are free of edges and holes out to the
maximum search radius.

We represent the geometry of the sources using the hierarchical pixel scheme
SDSSPix 
\footnote{http://lahmu.phyast.pitt.edu/$\sim$scranton/SDSSPix} 
\citep{Scranton03}, 
modeled after a similar scheme developed for CMB analysis
\citep{Gorski98}. This scheme represents well the rectangular
geometry of the SDSS stripes.  The mask is a collection of pixels at varying
resolution covering regions with holes or edges.  We do not mask out bright 
stars (which cover only a tiny fraction of the survey area) 
for the lensing analysis.

After checking the spherical polygon masks (\S
\ref{data:spectro:sphpolymasks}), each lens galaxy is checked against the pixel
mask to guarantee that it is within the allowed region.  Each quadrant around the
lens is then checked to determine if it contains a hole or an edge.  Lenses are
excluded from the sample 
if there are no adjacent quadrants that are completely unmasked. 
In addition to this cut, we demand that the
angular distribution of source galaxies 
around the lens have ellipticity no greater than
20\% in order to ensure the availability of pairs with 90\degr\ separation.
The same criteria are applied to the random points (see \S
\ref{gglmeas:random}).

We draw the final spectroscopic data set from SDSS stripes $9-12, 28-37$. 
We do not use galaxies from the 3 southern stripes
(76,82,86), because they contain few lens-source pairs at large separation.
After applying the cuts described above, the final lens sample contains \numspec\ 
galaxies.



\section{Results: The Mean \deltasig\ from $.02-10\MakeLowercase{h}^{-1}$ M\MakeLowercase{pc}} \label{gglmeas:deltasig}

The mean lensing signal \deltasig\ for the full 
sample of SDSS lens galaxies is shown in figure
\ref{fig:deltasig}.  The signal is unambiguously detected from 20 $h^{-1}$ kpc
to 10 $h^{-1}$Mpc.  Corrections have been made to this profile as described in
\S \ref{gglmeas:random}. These corrections are relatively small in all radial
bins.

\begin{figure}[htbp]
\plotone{deltasig.eps} 
\figcaption{ Mean \deltasig\ = $\overline{\Sigma}(<R) - \overline{\Sigma}(R)$
measured for the full lens sample. The solid line is the best-fitting power
law \deltasig$\varpropto R^{-0.76}$.
\label{fig:deltasig} }
\end{figure}

The errors in figure \ref{fig:deltasig} come from jackknife
re-sampling.  Although we expect statistical errors to 
dominate over sample variance
even on the largest scales shown here, there are in addition 
large variations in the systematic errors. 
Due to gaps between the 5 CCD columns, two interleaving 
imaging runs make up a contiguous imaging stripe, and they are generally 
taken on different nights under different photometric conditions. As a 
result, 
the residuals from the PSF correction vary between the columns of
interleaving runs, each of which is $\sim 0.2$ degrees wide.  
The residuals also
vary over time along the direction of the scan, with a typical scale of a few degrees.
Thus the proper subsample size to account for this variation is about a square
degree.  We divide the sample into 2,000 statistically equivalent ***what does 
this mean? don't you want to say 2000 disjoint subsamples?***
subsamples, each approximately a square degree in size (see \S
\ref{data:spectro:sphpolymasks}) and remeasure \deltasig\ 2,000 times, 
leaving out each subsample in turn. Figure \ref{fig:corr_matrix}
shows an image of the resulting correlation matrix 
$C_{i,j} = 
V_{i,j}/\sqrt{V_{i,i}V_{j,j}}$, where $V_{i,j} = \langle (\Delta \Sigma(R_i) - 
\langle \Delta \Sigma(R_i)\rangle)(\Delta \Sigma(R_j)-\langle \Delta \Sigma(R_j)
\rangle)\rangle$. 
The off-diagonal terms are negligible in the
inner bins but become important beyond $R \sim 1 h^{-1}$ Mpc.  The full covariance
matrix is used for all model fitting.  

***In this figure, you should convert ``radial bins'' to Mpc or at least 
say what the conversion is.***

\begin{figure}[bp]
\plotone{corr_matrix.eps} \figcaption{
Correlation matrix for \deltasig\ in figure \ref{fig:deltasig} calculated
using jackknife re-sampling. Each pixel number corresponds to a radial bin.
\label{fig:corr_matrix} }
\end{figure}

The mean \deltasig\ for the full sample is well described by a power law 
\begin{equation}
\Delta\Sigma(R) = A R^{-\alpha}
\end{equation}
with $\alpha = 0.76 \pm 0.05$. The outliers at intermediate radii make the
reduced $\chi^2$ for the power law fit 
somewhat poor but not unacceptable; there is a 20\% chance of
this $\chi^2$ occurring randomly (see table \ref{tab:allsamp}). 

An important check for systematic errors is the ``B-mode'', 
\deltacross\, the average shear signal 
measured at 45\degr\ with respect to the tangential component. 
If the tangential shear signal is due solely 
to lensing, the B-mode should be zero, whereas
systematic errors generally contribute to both the E- and B-modes.  The top panel of
figure \ref{fig:rand_and_ortho} shows \deltacross\ measured using the same 
source-lens sample used in figure \ref{fig:deltasig}. This measurement
is consistent with zero.

\begin{figure}[bp]
\plotone{rand_and_ortho.eps} \figcaption{
Two tests for systematics in the lensing measurement. The top panel shows the
``B-mode'' for lensing, \deltacross, measured around the same lenses used 
in figure \ref{fig:deltasig}.  The measurement is consistent with zero, as
expected for lensing. The bottom panel shows \deltasig\ measured around random
points.  The detection at large radius is indicative of systematic errors,
most likely from residuals in the PSF correction.  This has been subtracted
from the signal around lenses for figure \ref{fig:deltasig}. 
\label{fig:rand_and_ortho} }
\end{figure}

\subsection{Systematics Tests with Random Points} \label{gglmeas:random}

By replacing the lens 
galaxies with sets of random points, we can gauge two 
systematic effects on \deltasig: residuals
in the PSF correction and the radial bias due to clustering
of sources with the lenses. We generate random samples with ten times as many points
as the lens samples, using the same masks
and selection criteria described in \S \ref{data:spectro:sphpolymasks} and \S
\ref{data:spectro:pixelmasks}. The random points are also assigned the 
same redshifts ***or same redshift distribution?*** as the lens galaxies. 
These criteria guarantee that the same regions, and thus roughly the same
systematics, are sampled by the lenses and the random points.  Any non-zero
lensing signal for the random points we ascribe to residuals in the PSF correction.

The bottom panel of figure \ref{fig:rand_and_ortho} shows \deltasig\ measured
around \numrand\ random points, with errors from jackknife re-sampling. 
Note that the random sample is large enough that in this case sample variance
dominates the error in the outer bins. There is a significant signal at large radius. 
The signal at smaller radii is less 
well determined, but it is in any case far below the signal due to lenses.
Interpreting the large-scale signal as systematic error, we subtract it from
the \deltasig\ measured around lenses and add the errors in quadrature; this
correction is incorporated in figure \ref{fig:deltasig}. 

The second systematic probed by the random sample involves clustering of 
the source galaxies with the lenses. The calculation of 
the mean inverse critical density in \S \ref{lensing:deltasig} 
properly corrects for the fact that a fraction of the source galaxies 
are in front of the lenses, but only under the assumption that the lens and 
source galaxies are homogeneously distributed. Since galaxies are 
clustered, a small fraction of the sources are in fact physically 
associated with the lenses, causing a scale-dependent bias of the lensing 
signal. 
We correct for this by estimating the excess of sources around lenses 
compared with the random points.
The correction factor is the ratio of the sums of the weights for sources
around lenses and around random points:
\begin{equation} \label{eq:cluster_corr}
C(R) = \frac{N_{rand}}{N_{lens}} 
\frac{\sum_j \sigma_j^{-2}}{ \sum_i \sigma_i^{-2}}
\end{equation}
where $j$ indicates sources found around lenses, $i$ indicates 
sources found around random points, and $\sigma$ is the error in \deltasig\
for that source galaxy (see \S \ref{lensing:deltasig}) ***not sure 
what that means--is this the same weight as defined above?***. 
The value of $C(R)-1$ is shown for the full sample in figure
\ref{fig:cluster_corr}.  
The use of photometric redshifts reduces the correction significantly, since
sources associated with the lenses are given little weight.  The signal in
figure \ref{fig:deltasig} has been multiplied by $C(R)$.

\begin{figure}[bp]
\plotone{cluster_corr.eps} \figcaption{ Correction factor for the clustering of
sources around the lens galaxies.  The function $C(R)-1$ is essentially a
weighted cross-correlation function between lenses and sources.  The
\deltasig\ in figure \ref{fig:deltasig} has been multiplied by $C(R)$, which
is a negligible correction for radii larger than $\sim$ 50 kpc.
\label{fig:cluster_corr} }
\end{figure}

We measure 
this correction factor separately for each of the lens subsamples
presented in later sections.  Since galaxy clustering increases with luminosity,
and higher luminosity lens galaxies are seen out to 
higher redshift where more of the faint
sources are near the lenses, the correction factor $C(R)$  
increases with the luminosity of the lenses.  The correction for our highest
luminosity samples are a factor of ten larger than the correction for the
full sample. Similarly, the correction for early type lens galaxies
is larger, while the
correction for late types is smaller than for the full sample.  
Although the corrections for some
samples are large, the value of the correction is well measured in each
case.

We have also calculated the average 
\deltasig\ for the luminosity subsamples defined below (\S \ref{gmcflum}).  
Because $C(R)$ depends on luminosity, the average of the subsamples 
could in principle differ from the mean \deltasig\ estimated from the full sample 
\citep{Guzik02}.  However, we find no significant difference in the
two methods, another indication that the correction $C(R)$ is small with 
photozs.   

***Need some words in here about amplification effects: the correction above 
presumably overcorrects since some fraction of the source-lens association 
is due to amplification, not clustering. Or does it go the other 
way (anti-correlation)?***


\section{The Galaxy-mass Correlation Function} \label{gglmeas:wgm}

As noted in 
\S \ref{lensing:shear}, 
a power-law \deltasig\ is consistent, within the errors, with the 
galaxy-mass correlation function $\xi_{gm}$ also being a power-law, 
$\xi_{gm} = (r/r_0)^{-\gamma}$, with slope $\gamma = 1+\alpha$. 
The $\Delta \chi^2$ surface of the joint fit for $r_0$ and $\gamma$ is shown in
figure \ref{fig:wgm_fit_surf}. We find best fit marginalized values of $\gamma
= 1.76 \pm 0.05$ and $r_0 = 6.1 \pm 0.7 h^{-1}$ Mpc. These results assume the value 
for the mean density $\overline{\rho}=\rho_{crit} \Omega_m$ 
determined by WMAP + ACBAR + CBI in combination 
with the power spectrum from 2dFGRS and Lyman
$\alpha$ data, which yields $\Omega_m = 0.27 \pm 0.02$ \citep{Spergel03}. 
Marginalizing over this small uncertainty in $\Omega_m$ does not change our 
error estimates.
These results are summarized in Table \ref{tab:allsamp}.

***Here is where we should insert another sentence on the $\Omega_m$ 
dependence of the results: what if the true value is 0.2 or 0.4?***

We can compare these results with the galaxy auto-correlation function. 
\citet{Zehavi03} (hereafter Z03) analyzed $\xi_{gg}$ for a 
flux limited sample of 118,000 SDSS ``main'' galaxies, finding  
a best-fit power law with $\gamma = 1.80 \pm 0.02$ and
$r_0 = 5.77 \pm 0.14 h^{-1}$ Mpc. The selection function for the 
Z03 sample is very similar to that of our lens sample, but the 
former covers a more restricted range of absolute magnitudes: 
$-22.2<~M_r~<-18.9$, while our sample spans the range $-24
<~M_r~< -17$. However, Figure \ref{fig:absmag_hist} shows that a 
very large fraction of the galaxies in our lens sample in fact 
lie within the Z03 magnitude range.  
Also, while each galaxy is weighted differently in the two 
correlation function measurements, 
the mean luminosities of the two samples are quite similar:  
for the Z03 sample, $\langle L_r \rangle =
1.7\times 10^{10} h^{-2} L_{\sun}$, while the mean for our sample is
$\langle L_r \rangle =1.45\times 10^{10} h^{-2} L_{\sun}$, both 
comparable to $L_* = 1.54\times 10^{10} h^{-2} L_{\sun}$.  
Assuming that $r_0$ for $\xi_{gg}$ scales as
in \citet{Norberg01}, we expect the auto-correlation length for our sample of
galaxies to be only slightly lower than that of Z03, $r_0 = 5.60$ rather than
$5.77 h^{-1}$ Mpc. 



\begin{figure}[tp]
\plotone{wgm_fit_surf.eps} \figcaption{Contours in $\Delta \chi^2$ for a power
law fit to \deltasig\ for the full sample, 
with $\xi_{gm} = (r/r_0)^{-\gamma}$ and assuming $\Omega_m =
0.27$.  The contours enclose the joint 1-, 2-, and 3-$\sigma$ regions.  The
best-fit values are $\gamma = 1.76 \pm 0.05$ and $r_0 = 6.1 \pm 0.7$
(1-$\sigma$).
\label{fig:wgm_fit_surf} }
\end{figure}


%In figure \ref{fig:wgm_wgg} we show \wgm\ based on the best-fit power law and
%$\Omega_m = 0.27$.  This figure, which is just meant for illustration, is
%simply \deltasig\ scaled by the factor shown in equation \ref{deltasig2wgm};
%the errors have not been scaled to represent the true uncertainty in the
%normalization.  Also over-plotted is the galaxy auto-correlation function
%$w_{gg}$ from the SDSS \citep{Zehavi03}.  We will have more to say about this
%comparison and its implications for bias in section \S \ref{gmcfcompare}

\begin{figure}[bp]
%\plotone{wgm_wgg.eps} 
\figcaption{Galaxy-mass correlation function found by inverting 
\deltasig\ (assuming $\Omega = 0.27$) compared with 
the galaxy autocorrelation function for a similarly selected 
sample of SDSS galaxies.
\label{fig:wgm_wgg} }
\end{figure}

In addition to fitting a power law to 
the lensing measurements, we can directly invert them to find $\xi_{gm}(r)$ 
(\S \ref{lensing:shear}). 
These results are shown in Figure \ref{fig:wgm_wgg} and 
compared with the galaxy auto-correlation 
function $\xi_{gg}$ measured by Z03.
The $\xi_{gg}$ and $\xi_{gm}$ for these samples are clearly consistent 
within the errors over the common range of radii.
The galaxy-mass bias, defined in equation \ref{bgmdef}, is shown 
in Figure ZZZ; it is consistent with unity and is scale
independent within the errors. 
We find a mean bias from $0.1-10 h^{-1}$ Mpc of
\begin{equation}
\langle b_{gm} \rangle = 0.9 \pm 0.2~.
\end{equation}


\section{Dependence of Galaxy-mass Correlations on Luminosity}
\label{gmcflum}

To study galaxy-mass clustering as a function of galaxy luminosity, we 
split the lens sample into three bins of luminosity separately in each of the
five SDSS bandpasses, as shown in Table
\ref{tab:lumbin}. The binning is chosen primarily to yield comparable 
lensing S/N in each bin. To achieve this we place 85\% of the 
galaxies in the lowest luminosity bin, 10\% in the middle, and 
5\% in the highest bin. ***need a sentence about why the numbers 
differ slightly from band to band***
The low-luminosity bin must be so large because 
intrinsically faint galaxies are weakly 
clustered and are only detected in the SDSS at low redshift (for 
which the lensing efficiency is low). Since it contains 
the majority of lens galaxies, the 
faintest subsample has a mean luminosity comparable to $L_*$, while 
the brightest subsample is $\sim 5-7$ times brighter. 





\begin{figure*}[tbp]
\plotone{deltasig_all_allband_bylum_color_fits.eps} \figcaption{ Mean
\deltasig\ in three luminosity subsamples for each of the 5 SDSS bandpasses.  In
each panel, filled circles (black),
crosses (blue), and triangles (red) 
represent measurements for the lowest, middle, and highest 
luminosity subsamples. The data have been re-binned to 9 bins from 18 for
clarity. Also shown are the best fit power laws for each subsample. 
\label{fig:deltasig_allband_bylum} }
\end{figure*}

The mean \deltasig\ for each luminosity subsample is shown in Figure
\ref{fig:deltasig_allband_bylum}. The profiles
have been re-binned from 18 to 9 radial bins for clarity of presentation.
Note that, although the
different bins in luminosity are independent, the subsamples in the different
bandpasses are simply re-samplings of the same lenses.
In each bandpass, the slope of \deltasig\ for the lowest luminosity bin 
is smaller than those of the two
higher bins. 

We fit the power law model of equation \ref{xigm_def} to each of the
subsamples.  For the lowest luminosity subsample in each bandpass, 
we use all the radial bins since \deltasig\ is consistent with a 
single power law. Since 
the \deltasig\ profiles for the higher luminosity bins have a 
flatter inner slope, in those cases we use
only measurements at $R > 100$ kpc for the power law fits.  The results for these
fits are shown in Table \ref{tab:lumbin}, and the associated 
$\Delta \chi^2$ contours are shown in Figure \ref{fig:deltasig_allband_bylum_fits}.

Since the lowest luminosity subsample contains the large majority of lenses, 
it is not surprising that the slope and amplitude of $\Delta \Sigma$ 
in this case are consistent with the values for the full sample given in 
Table \ref{tab:allsamp}.  The higher
luminosity subsamples show a significantly steeper slope, $\gamma > 2.0$, 
that is equal within the errors for these two subsamples. 
Within these two higher bins the scale length $r_0$ increases with
luminosity.  These trends of slope and amplitude are seen for all 5
bandpasses (Fig. \ref{fig:deltasig_allband_bylum_fits}). 

For bright 
galaxies, a qualitatively similar trend is seen 
in the galaxy auto-correlation function \citep{Norberg01,Zehavi02}: the 
correlation length increases with luminosity while the slope remains 
roughly constant. Quantitative comparison is difficult, however, 
since we use flux-limited lens galaxy samples (due to the 
limited S/N), while the galaxy auto-correlation studies use volume-limited samples. 
It is less clear whether the change in $\xi_{gm}$ slope we see 
between $L \sim L_*$ and $L \sim 3L_*$ galaxies is seen in the 
auto-correlation function. 

A priori, one might hypothesize 
that the change in \deltasig\ slope from $L_*$ to brighter galaxies 
is partly due to the fact that the 
higher luminosity subsamples are also redder: for the \rmag-band 
samples, $g-r = 0.73(0.77)$ 
for the middle(brightest) sample, while $g-r = 0.63$ for the full sample. 
However, the same increase of slope is also seen for the \umag\ subsamples, 
for which the trend of $g-r$ color with luminosity is much less pronounced.


\begin{figure*}[tbp]
\plotone{deltasig_chisq_all_allband_bylum_color_fits.eps} \figcaption{ Same as
figure \ref{fig:wgm_fit_surf} but the confidence contours are shown separately
for the three luminosity subsamples in each of the five SDSS bandpasses: 
solid (black), dotted (blue), and dot-dashed (red) contours 
represent fits from lowest to highest luminosity.
\label{fig:deltasig_allband_bylum_fits} }
\end{figure*}

\section{Dependence of Galaxy-mass Correlations on Galaxy Spectral Type and Color}
\label{gmcftype}

%\subsubsection{Galaxy Spectral Classification, Color, and Velocity Dispersion} 
%\label{data:spectro:class}

In addition to luminosity, we can also study how the lensing signal varies 
with galaxy spectral type and color.
The \spectroone\ pipeline classifies galaxies as early to late spectral types using 
the method developed in
\citet{Connolly95} and \citet{Connolly99}.  A Karhunen-Lo\`{e}ve decomposition
(KL) is performed on a large set ($\sim 100,000$) 
of galaxy spectra, ranging from quiescent
E-Sc types to starburst irregulars. 
The spectrum of each galaxy is expanded in terms of the KL eigenspectra, and the 
first five 
coefficients of the expansion \{\texttt{ECOEFF1,..,ECOEFF5}\} are measured. 
Since most of the information is contained 
in the first two coefficients,
a simple one-parameter family constructed from the angle in the plane
of the first two eigenvectors is used to define spectral type: \eclass\ =
$\tan^{-1}(-\texttt{ECOEFF2/ECOEFF1})$.  Early types have more negative
\eclass.  

The distribution of \eclass\ for the lens galaxy sample is shown in the bottom
panel of figure \ref{fig:eclass}.  The rest frame \gmr\ distribution is also
shown in the top panel.  The \gmr\ color and \eclass\ correlate well.  We make
cuts at \eclass=\eclasscut\ and \gmr\ = 0.7 to split the sample into 
early/late types and red/blue.  These cuts each divide the full 
sample roughly in half.  

\begin{figure}[tbp]
%\centering
%\includegraphics[width=240pt]{eclass_dist.eps}
%\caption{
\plotone{eclass_gmr_hist.eps} \figcaption{ Distribution of \gmr\ (top panel)
and \eclass\ (bottom panel) for lens galaxies.  
Early types have more negative \eclass, which has
been plotted in reverse for comparison with \gmr. We divide the
samples at \gmr\ = \gmrcut\ and \eclass=\eclasscut\ as shown by the gray
vertical lines. \label{fig:eclass}}
\end{figure}


The mean \deltasig\ for these different classes is shown in Figure
\ref{fig:deltasig_early_late}.  The red galaxies show a power law 
$\Delta \Sigma$ for $R > 100$
kpc and a flattening at smaller separations, similar to the two highest
luminosity bins of the full sample.  The blue and late type measurements 
are consistent with a power law \deltasig\ over all separations.

We fit a power law galaxy-mass 
correlation function to each sample, using only $R > 100$
kpc for the early and red samples. The results are shown in Table
\ref{tab:allsamp}.  The red/early type galaxies have a higher amplitude than
the blue/late types, as indicated by the best-fit correlation length $r_0$. The 
red/early subsamples are also nearly twice as luminous as the blue/late subsamples.
samples also have higher mean luminosities. We detect no appreciable 
difference in the
slope of the blue/late correlation function relative to that of 
the red/early types, in contrast to the strong shift in 
slope of the galaxy auto-correlation function as a function of 
type and color \citep{Zehavi02}. On the other hand, the errors in 
the blue/late sample slopes are sizeable, reflecting the lower S/N. 

\begin{figure}[t]
\plotone{deltasig_early_late_sidebyside_color_fits.eps} \figcaption{ Mean
\deltasig\ for different classes of galaxies. The left panel shows \deltasig\
for galaxies classified as early (closed symbols; red) 
and late (open symbols; blue) spectral types (\eclass$>$
\eclasscut). Right panel shows a split by \gmr\ color.  The two 
trends are very similar, as expected since \gmr\ and \eclass\ are highly 
correlated.
\label{fig:deltasig_early_late} }
\end{figure}

\subsection{Dependence of $\xi_{gm}$ on Luminosity for Red Galaxies} 
\label{gmcfredlum}

The lensing S/N for the red galaxy sample is quite high, 
allowing us to further subdivide this sample into three bins of
luminosity in each of the 5 sdss bandpasses, as we did in \S \ref{gmcflum}.
The split is again 85\%, 10\%, and 5\% of the galaxies in each 
of the three luminosity bins. 
The mean \deltasig\ for these luminosity bins is shown in Figure
\ref{fig:deltasig_red_allband_bylum}.  The trends of \deltasig\ with 
red galaxy luminosity are similar to those for the full 
sample.  The fits to power law correlation functions are listed in
Table \ref{tab:earlylumbin}, and the $\Delta \chi^2$ surfaces are shown in
Figure \ref{fig:deltasig_red_allband_bylum_fits}. Note that the 
correlation length $r_0$ and mean luminosity 
for each red luminosity subsample is larger than that of the 
corresponding full luminosity subsample, as expected from 
\S \ref{gmcftype}. 

We now have two pieces of evidence suggesting that the increase 
in clustering strength we observe is more closely associated with 
luminosity than with color: (i) for the subsamples split only by 
luminosity (not color), galaxies split by \umag-band luminosity 
show a significant increase of clustering strength with luminosity, 
even though the mean $g-r$ color is nearly independent of luminosity for 
these subsamples; (ii) within the red galaxy sample itself, we find 
the same increase of clustering strength with luminosity that we 
see for the full sample.

\begin{figure*}[tp]
\plotone{deltasig_redthree_allband_bylum_color_fits.eps} \figcaption{ 
Same as figure \ref{fig:deltasig_allband_bylum} but only for red galaxies, 
\gmr\ $ > $ \gmrcut.
\label{fig:deltasig_red_allband_bylum} }
\end{figure*}

\begin{figure*}[bp]
\plotone{deltasig_chisq_redthree_allband_bylum_color_fits.eps} \figcaption{
Same as figure \ref{fig:deltasig_allband_bylum_fits}, but only for red galaxies, 
\gmr\ $ > $ \gmrcut.  
\label{fig:deltasig_red_allband_bylum_fits} }
\end{figure*}

\subsection{Dependence of \deltasig\ on Velocity Dispersion}  \label{gmcfvdis}

For a large fraction of the early type galaxies, selected by 
\eclass\ and by the requirement that the surface brightness approximately 
fits a de Vaucouleurs profile, the SDSS spectroscopic pipeline measures 
the velocity dispersion $\sigma_v$. The 
distribution of velocity dispersions is shown in figure \ref{fig:vel_dis_hist}.
We use only galaxies with $50 < \sigma_v ({\rm km/sec}) < 400$ for this study and 
split the sample roughly in half at $\sigma_v = 182$ km/sec.

***We should state the mean $\sigma_v$ values for 
each of these two subsamples.***



\begin{figure}[bp]
\plotone{vel_dis_hist.eps} \figcaption{
Histogram of measured velocity
dispersion $\sigma_v$ for \numvdis\ early-type lens galaxies.
\label{fig:vel_dis_hist}}
\end{figure}



In Figure
\ref{fig:deltasig_vdis} we show 
\deltasig\ for these two samples of 
early-type galaxies split by velocity dispersion.  
The half of the sample 
with higher $\sigma_v$ shows 
a steeper slope, and a larger amplitude at $R < 1 h^{-1}$ Mpc, 
than the low-$\sigma_v$ galaxies. For 
neither sample is \deltasig\ well fit by a power law.
Given the Faber-Jackson relation between velocity dispersion and 
luminosity, this trend agrees qualitatively with the results above 
for the luminosity-scaling of \deltasig\ for red galaxies. 




\begin{figure}[tp]
\plotone{deltasig_vdis_color.eps} \figcaption{ 
Mean \deltasig\ for early type galaxies with measured velocity dispersion 
$\sigma_v > 182$ km/sec (red/dashed curve) and $< 182$ km/sec (blue/solid curve).
The slope and amplitude increase with $\sigma_v$, as expected from 
the trend with luminosity.
\label{fig:deltasig_vdis} }
\end{figure}


%Other than the conversion to \wgm, which requires the use of $\Omega_m$, our
%results are not sensitive to the assumed cosmology.  This is because the
%primary dependence is in the angular diameter distances, which are insensitive
%to cosmology at low redshift.  We repeated our measurements assuming $\Omega_m
%= 1$ and the results changed by .

\section{Discussion} \label{discussion}

We have presented measurements of the 
mean lensing signal \deltasig\ for a flux limited spectroscopic sample of SDSS
galaxies with $\langle L \rangle \sim L_*$ and $\langle z \rangle 
\simeq 0.1$ over scales 
20$h^{-1}$ kpc to 10$h^{-1}$ Mpc.  The results are consistent with 
a power-law galaxy-mass correlation function over this  
range, $\xi_{gm} = (r/r_0)^{-\gamma}$.  
The measured slope, $\gamma =1.76 \pm 0.05$, is 
consistent with that of the galaxy auto-correlation
function $\xi_{gg}$ for a similarly selected set of galaxies
\citep{Zehavi03}. This implies that the bias parameter 
$b_{gm} = \xi_{gg}/\xi_{gm} 
=b/r$ is scale independent over this range, with amplitude 
$b_{gm} = (0.9 \pm 0.2)(\Omega_m/0.27)^{1/2}$. 
Semi-analytic galaxy formation models suggest that
$b_{gm}$ is a direct measure of the standard bias $b$, i.e., 
that $r$ approaches unity, for separations larger
than 1 Mpc \citep{Guzik01}, although there are hints from 
other lensing data that this
may not be the case \citep{Hoekstra02b}.

These results are consistent with and extend previous galaxy 
lensing measurements which did not incorporate lens galaxy redshifts. 
\citet{fis00} found a power law \deltasig\ for angular
scales $\theta < 600$\arcsec, or roughly $R < 1 h^{-1}$ Mpc. 
\citet{Wilson01} and \citet{Hoekstra03a} found similar
results over a smaller range of scales, although the power law index was less
well constrained.  \citet{Hoekstra02b} used a deeper, narrower 
sample than ours to obtain a high S/N measurement of $\xi_{gm}$ over effective 
scales $R< 5 h^{-1}$ Mpc. 
They find that $b/r$ is approximately scale-dependent over 
the range $0.1-3 h^{-1}$ Mpc, with mean amplitude 
$b_{gm} = 1.09 \pm 0.04$ over these
scales. However, combining their galaxy lensing data with deeper 
cosmic shear data, they find evidence that $b$ and $r$ separately 
vary with scale.

Splitting our lens galaxies into subsamples has enabled us to 
study $\xi_{gm}$ as a function of luminosity, color, and spectral type. 
The amplitude of $\xi_{gm}$ increases 
with increasing luminosity. Moreover, on scales $R> 0.1 h^{-1}$ Mpc 
the slope of $\xi_{gm}$ for 
the two brighter subsamples, with $L \sim 3$ and $4.5 L_*$, is 
steeper than that for the $L \sim L_*$ galaxies. 
Within the 
two brighter subsamples, the large-scale slope is approximately constant but 
the amplitude increases with luminosity.
These trends of the slope and amplitude of $\xi_{gm}$ with 
luminosity are in good 
qualitative agreement with the predictions of semi-analytic 
galaxy formation models \citep{Guzik01} on scales $r < 1 h^{-1}$ Mpc. 
On the other hand, their semi-analytic models predict that $\xi_{gm}$ 
is independent of luminosity for scales $r > 1 h^{-1}$ Mpc, 
in disagreement with our results.  
The increase in clustering strength with luminosity we observe 
is also similar to that seen for the
galaxy-autocorrelation function \citep{Norberg01,Zehavi02}, but the steepening
at higher luminosity 
has not been seen in previous autocorrelation measurements. 
Newer measurements from the SDSS,
however, do show an increase in slope with luminosity \citet{Zehavi04}.

On smaller scales, $R< 100 h^{-1}$ kpc, 
the slope for the brighter subsamples flattens. This may be 
a real effect, or it could be caused by the mass-sheet degeneracy.
Consider a galaxy embedded in a large dark halo.
Over small scales, the halo 
is approximately a uniform mass sheet and does not contribute to the 
lensing signal---\deltasig\
is only sensitive to the central galaxy.  Over larger scales, the 
halo density
varies within the aperture, and \deltasig\ is sensitive to
both the central galaxy and the halo.  For very luminous
galaxies, which tend to live in clusters, this effect may be strong.
A corresponding 
flattening of the galaxy auto-correlation function slope 
for bright galaxies on small scales has not been seen.


We have also detected a trend in $\xi_{gm}$ with galaxy color and 
spectral type. Red/early type galaxies show stronger correlation amplitude 
than blue/late types but the slopes of $\xi_{gm}$ for the two subsamples 
are equal within the errors. This behavior is also in qualitative 
agreement with the semi-analytic galaxy formation predictions 
\citep{Guzik01}.

Finally, we have sufficient S/N to study the scaling of $\xi_{gm}$ 
with luminosity and velocity dispersion for red/early type galaxies. 
Again, we find an increase in logarithmic slope and amplitude with 
increasing $L$ and $\sigma_v$. 

A separate analysis will compare these results in detail with 
predictions from N-body simulations and the halo model of structure 
formation.

%%%%%%%%%%%%%%%%%%%%%%%%%%%%
% Acknowledgements 
%%%%%%%%%%%%%%%%%%%%%%%%%%%%

\acknowledgments

We thank Andreas Berlind and Idit Zehavi for useful discussions and 
Idit Zehavi for providing the $\xi_{gg}$ measurements.

Add funding sources.

Funding for the creation and distribution of the SDSS Archive has been provided
by the Alfred P. Sloan Foundation, the Participating Institutions, the National
Aeronautics and Space Administration, the National Science Foundation, the
U.S. Department of Energy, the Japanese Monbukagakusho, and the Max Planck
Society. The SDSS Web site is http://www.sdss.org/.
The SDSS is managed by the Astrophysical Research Consortium (ARC) for the
Participating Institutions. The Participating Institutions are the University
of Chicago, Fermilab, the Institute for Advanced Study, the Japan Participation
Group, the Johns Hopkins University, Los Alamos National Laboratory, the
Max-Planck-Institute for Astronomy (MPIA), the Max-Planck-Institute for
Astrophysics (MPA), New Mexico State University, the University of Pittsburgh,
Princeton University, the United States Naval Observatory, and the University
of Washington.

\section{Appendix: Photoz Systematic Errors} \label{systematics}

We discussed two sources of 
systematic errors in \S \ref{gglmeas:random}: residuals
from the PSF correction and clustering of faint sources around the lens
galaxies.  These biases have been estimated from the data 
and corrected for in our final results.

Another possible source of error may come from biases in the photometric
redshifts (\S \ref{data:imaging:photoz}).  Although the photometric 
redshift distributions
inferred for SDSS galaxies are consistent with published redshift surveys for
\rmag$~<21$ \citep{Csabai2003}, there are well-known degeneracies 
in the technique (between type and redshift) that tend to 
push galaxies to particular values of photoz.  This pile-up 
can be seen in figure
\ref{fig:photozdist}: there are significant peaks in the photoz distribution
at several redshifts. Although such peaks are often seen in spectroscopic 
redshift surveys, where they are associated with large-scale structures, 
the volume probed by the SDSS photoz sample is so large, 
containing $\sim 4$ million galaxies covering $\sim 1000$ 
square degrees out to $z \sim 0.6$,  that the effects of such 
structures should be very small. 

Because galaxies in parameter regions 
with strong degeneracies have relatively large photoz errors,
we expect this bias to be suppressed in the final analysis. We have repeated
our analysis using only the source galaxy 
redshift prior inferred from summing the individual
Gaussians (as shown in figure \ref{fig:photozdist}), i.e., not 
using the photoz measurements for individual source galaxies, 
and we recover the same
results within the noise.  Similarly, we have used the method outlined in
\citet{Mckay02}, which uses the \rmag\ magnitudes to infer the redshift
distribution, and recover consistent results. 
Thus the bias from photoz degeneracies appears to be small compared to our
statistical uncertainty.



\newpage

\bibliographystyle{apj}
% Bib database
\bibliography{apj-jour,astroref}

%%%%%%%%%%%%%%%%%%%%%%%%%%%%%%%
% Figure captions here
%%%%%%%%%%%%%%%%%%%%%%%%%%%%%%%

\newpage

%%%%%%%%%%%%%%%%%%%%%%%%%%%%%%%
% Table for all/early,etc.
%%%%%%%%%%%%%%%%%%%%%%%%%%%%%%%

\begin{deluxetable}{cccccccc}
\tabletypesize{\small}
\tablecaption{Correlation Functions for All, Early, and Late Type Galaxies \label{tab:allsamp}}
\tablewidth{0pt}
\tablecomments{Absolute magnitudes are \rmag-band Petrosian $M - 5 \log_{10} h$. 
               Values in parentheses are luminosity in units of $10^{10} h^{-2} L_{\sun}$.
               The means are calculated using the same weights as the lensing measurement.
               The value of $M_* (L_*)$ for the \rmag-band is -20.83 (1.54).
               The $r_0$ and $\gamma$ are best fit parameters for $\xi_{gm} = 
               (r/r_0)^{-\gamma}$; $r_0$ is measured in $h^{-1}$ Mpc. A value of
               $\Omega_m = 0.27$ was assumed. The outer 13 bins were used for the ``early'' and ``red'' sample fits.  For the ``late'' and ``blue'' samples, the data were rebinned from 18 to 9 radial bins.}
\tablehead{
\colhead{Sample} &
\colhead{Selection Criteria} &
\colhead{Mean Abs. Mag.} &
\colhead{Mean \gmr} &
\colhead{N$_{Lenses}$} &
\colhead{$r_0$} &
\colhead{$\gamma$} &
\colhead{$\chi^2/\nu$} 
}
\
\startdata
 All & - & -20.767 (1.455 $\pm$ 0.004) & 0.629 & 127001 & 6.1 $\pm$ 0.7 & 1.76 $\pm$ 0.05 & 20.2/16 \\
 Red & \gmr\ $ > $ \gmrcut\ & -21.061 (1.908 $\pm$ 0.006) & 0.753 & 60099 & 6.7 $\pm$ 1.0 & 1.85 $\pm$ 0.07 & 11.1/11 \\
 Blue & \gmr\  $ < $ \gmrcut\ & -20.477 (1.114 $\pm$ 0.004) & 0.536 & 65134 & 4.6 $\pm$ 1.0 & 1.71 $\pm$ 0.11 & 5.72/7 \\
 Early & \eclass\ $ < $ \eclasscut\ & -21.035 (1.862 $\pm$ 0.006) & 0.737 & 62340 & 6.8 $\pm$ 1.0 & 1.84 $\pm$ 0.06 & 13.7/11 \\
 Late & \eclass\ $ > $ \eclasscut\ & -20.473 (1.110 $\pm$ 0.004) & 0.539 & 64378 & 3.9 $\pm$ 0.9 & 1.79 $\pm$ 0.12 & 7.64/7
\enddata
\end{deluxetable}


%%%%%%%%%%%%%%%%%%%%%%%%%%%%%%%%%%%%%%%%%
% Table for luminosity bins
%%%%%%%%%%%%%%%%%%%%%%%%%%%%%%%%%%%%%%%%%

\begin{deluxetable}{cccccccc}
\tabletypesize{\small}
\tablecaption{Luminosity Bins for All Galaxies \label{tab:lumbin}}
\tablewidth{0pt}
\tablecomments{See table \ref{tab:allsamp} for explanation of columns. The 
               value of $M_* (L_*)$ is -18.34(0.77), -20.04(1.10), 
               -20.83(1.54), -21.26(2.07), -21.55(2.68) for $u, g, r, i, z$ 
               respectively.} 
\tablehead{
\colhead{Bandpass} &
\colhead{Abs. Mag. Range} &
\colhead{Mean Abs. Mag.} &
\colhead{Mean \gmr} &
\colhead{N$_{Lenses}$} &
\colhead{$r_0$} &
\colhead{$\gamma$} &
\colhead{$\chi^2/\nu$} 
}
\
\startdata
$u$ & -19.6 $ < M_{u} < $ -15.0 & -18.515 (0.908 $\pm$ 0.002) & 0.628 & 106585 & 6.2 $\pm$ 0.9 & 1.72 $\pm$ 0.06 & 15.6/16 \\
 -  & -20.0 $ < M_{u} < $ -19.6 & -19.763 (2.864 $\pm$ 0.003) & 0.639 & 12539 & 3.8 $\pm$ 0.9 & 2.16 $\pm$ 0.13 & 6.80/11 \\
 -  & -22.0 $ < M_{u} < $ -20.0 & -20.318 (4.779 $\pm$ 0.020) & 0.653 & 6270 & 6.1 $\pm$ 1.1 & 2.11 $\pm$ 0.09 & 13.7/11 \\
    & & & & & & \\
$g$ & -21.0 $ < M_{g} < $ -16.5 & -19.891 (0.956 $\pm$ 0.002) & 0.623 & 106646 & 6.0 $\pm$ 0.9 & 1.71 $\pm$ 0.06 & 17.0/16 \\
 -  & -21.4 $ < M_{g} < $ -21.0 & -21.188 (3.156 $\pm$ 0.004) & 0.699 & 12546 & 4.3 $\pm$ 0.9 & 2.16 $\pm$ 0.11 & 6.25/11 \\
 -  & -23.5 $ < M_{g} < $ -21.4 & -21.718 (5.145 $\pm$ 0.025) & 0.735 & 6275 & 6.2 $\pm$ 0.9 & 2.16 $\pm$ 0.07 & 12.6/11 \\
    & & & & & & \\
$r$ & -21.7 $ < M_{r} < $ -17.0 & -20.544 (1.184 $\pm$ 0.003) & 0.621 & 106643 & 6.1 $\pm$ 0.9 & 1.71 $\pm$ 0.06 & 17.6/16 \\
 -  & -22.2 $ < M_{r} < $ -21.7 & -21.902 (4.137 $\pm$ 0.005) & 0.726 & 12543 & 4.4 $\pm$ 0.9 & 2.15 $\pm$ 0.11 & 10.4/11 \\
 -  & -24.0 $ < M_{r} < $ -22.2 & -22.463 (6.937 $\pm$ 0.026) & 0.768 & 6274 & 6.6 $\pm$ 0.9 & 2.15 $\pm$ 0.07 & 14.5/11 \\
    & & & & & & \\
$i$ & -22.0 $ < M_{i} < $ -17.0 & -20.870 (1.446 $\pm$ 0.003) & 0.620 & 106625 & 6.0 $\pm$ 1.0 & 1.71 $\pm$ 0.06 & 19.2/16 \\
 -  & -22.5 $ < M_{i} < $ -22.0 & -22.232 (5.067 $\pm$ 0.006) & 0.734 & 12544 & 3.9 $\pm$ 0.7 & 2.23 $\pm$ 0.11 & 9.29/11 \\
 -  & -24.0 $ < M_{i} < $ -22.5 & -22.776 (8.368 $\pm$ 0.030) & 0.773 & 6271 & 6.3 $\pm$ 0.8 & 2.17 $\pm$ 0.07 & 13.3/11 \\
    & & & & & & \\
$z$ & -22.2 $ < M_{z} < $ -17.0 & -21.069 (1.721 $\pm$ 0.004) & 0.619 & 105975 & 6.0 $\pm$ 1.0 & 1.70 $\pm$ 0.06 & 18.6/16 \\
 -  & -22.6 $ < M_{z} < $ -22.2 & -22.350 (5.599 $\pm$ 0.007) & 0.725 & 12466 & 4.5 $\pm$ 0.9 & 2.14 $\pm$ 0.10 & 10.5/11 \\
 -  & -24.0 $ < M_{z} < $ -22.6 & -22.881 (9.132 $\pm$ 0.032) & 0.754 & 6236 & 5.4 $\pm$ 0.8 & 2.21 $\pm$ 0.08 & 6.45/11
\enddata
\end{deluxetable}



% Not symmeric erors all
%$u$ & -19.6 $ < M_{u} < $ -15.0 & -18.515 $\pm$ 0.002 & 106585 & 6.2 $\pm$ 0.9 & 1.72$^{+0.05}_{-0.06}$ & 15.6/16 \\
% -  & -20.0 $ < M_{u} < $ -19.6 & -19.763 $\pm$ 0.001 & 12539 & 3.8 $\pm$ 0.9 & 2.16 $\pm$ 0.13 & 6.80/11 \\
% -  & -22.0 $ < M_{u} < $ -20.0 & -20.292 $\pm$ 0.003 & 6270 & 6.1 $\pm$ 1.0 & 2.11 $\pm$ 0.09 & 13.7/11 \\
%    & & & & & & \\
%$g$ & -21.0 $ < M_{g} < $ -16.5 & -19.891 $\pm$ 0.002 & 106646 & 6.0 $\pm$ 0.9 & 1.71 $\pm$ 0.06 & 17.0/16 \\
% -  & -21.4 $ < M_{g} < $ -21.0 & -21.188 $\pm$ 0.001 & 12546 & 4.3 $\pm$ 0.9 & 2.16 $\pm$ 0.11 & 6.25/11 \\
% -  & -23.5 $ < M_{g} < $ -21.4 & -21.699 $\pm$ 0.003 & 6275 & 6.2$^{+0.8}_{-0.9}$ & 2.16 $\pm$ 0.07 & 12.6/11 \\
%    & & & & & & \\
%$r$ & -21.7 $ < M_{r} < $ -17.0 & -20.544 $\pm$ 0.002 & 106643 & 6.1$^{+0.9}_{-1.0}$ & 1.71 $\pm$ 0.06 & 17.6/16 \\
% -  & -22.2 $ < M_{r} < $ -21.7 & -21.902 $\pm$ 0.001 & 12543 & 4.4$^{+0.8}_{-0.9}$ & 2.15$^{+0.10}_{-0.11}$ & 10.4/11 \\
% -  & -24.0 $ < M_{r} < $ -22.2 & -22.456 $\pm$ 0.004 & 6274 & 6.6$^{+0.8}_{-0.9}$ & 2.15 $\pm$ 0.07 & 14.5/11 \\
%    & & & & & & \\
%$i$ & -22.0 $ < M_{i} < $ -17.0 & -20.870 $\pm$ 0.002 & 106625 & 6.0$^{+0.9}_{-1.0}$ & 1.71 $\pm$ 0.06 & 19.2/16 \\
% -  & -22.5 $ < M_{i} < $ -22.0 & -22.232 $\pm$ 0.001 & 12544 & 3.9 $\pm$ 0.7 & 2.23 $\pm$ 0.10 & 9.29/11 \\
% -  & -24.0 $ < M_{i} < $ -22.5 & -22.774 $\pm$ 0.004 & 6271 & 6.3 $\pm$ 0.8 & 2.17 $\pm$ 0.07 & 13.3/11 \\
%    & & & & & & \\
%$z$ & -22.2 $ < M_{z} < $ -17.0 & -21.069 $\pm$ 0.002 & 105975 & 6.0$^{+0.9}_{-1.0}$ & 1.70 $\pm$ 0.06 & 18.6/16 \\
% -  & -22.6 $ < M_{z} < $ -22.2 & -22.350 $\pm$ 0.001 & 12466 & 4.5$^{+0.8}_{-0.9}$ & 2.14 $\pm$ 0.10 & 10.5/11 \\
% -  & -24.0 $ < M_{z} < $ -22.6 & -22.881 $\pm$ 0.004 & 6236 & 5.4 $\pm$ 0.8 & 2.21 $\pm$ 0.08 & 6.45/11


%%%%%%%%%%%%%%%%%%%%%%%%%%%%%%%%%%%%%%%%%
% Table for Early Type luminosity bins
%%%%%%%%%%%%%%%%%%%%%%%%%%%%%%%%%%%%%%%%%

\begin{deluxetable}{cccccccc}
\tabletypesize{\small}
\tablecaption{Luminosity Bins for Red Galaxies \label{tab:earlylumbin}}
\tablewidth{0pt}
\tablecomments{Same as table \ref{tab:lumbin} but for red galaxies, defined
               as galaxies with \gmr\ $ > 0.7$.}
\tablehead{
\colhead{Bandpass} &
\colhead{Abs. Mag. Range} &
\colhead{Mean Abs. Mag.} &
\colhead{Mean \gmr} &
\colhead{N$_{Lenses}$} &
\colhead{$r_0$} &
\colhead{$\gamma$} &
\colhead{$\chi^2/\nu$} 
}
\
\startdata
$u$ & -19.6 $ < M_{u} < $ -15.0 & -18.521 (0.913 $\pm$ 0.003) & 0.752 & 51074 & 7.8 $\pm$ 1.0 & 1.74 $\pm$ 0.05 & 12.7/16 \\
 -  & -20.1 $ < M_{u} < $ -19.6 & -19.791 (2.939 $\pm$ 0.005) & 0.769 & 6008 & 5.2 $\pm$ 0.8 & 2.19 $\pm$ 0.08 & 13.6/11 \\
 -  & -22.0 $ < M_{u} < $ -20.1 & -20.373 (5.026 $\pm$ 0.027) & 0.788 & 3006 & 8.2 $\pm$ 1.2 & 2.12 $\pm$ 0.07 & 16.4/11 \\
    & & & & & & \\
$g$ & -21.2 $ < M_{g} < $ -16.5 & -20.112 (1.171 $\pm$ 0.003) & 0.752 & 51084 & 7.9 $\pm$ 1.1 & 1.72 $\pm$ 0.05 & 11.9/16 \\
 -  & -21.6 $ < M_{g} < $ -21.2 & -21.364 (3.712 $\pm$ 0.006) & 0.773 & 6008 & 5.3 $\pm$ 0.8 & 2.19 $\pm$ 0.08 & 7.31/11 \\
 -  & -23.5 $ < M_{g} < $ -21.6 & -21.841 (5.762 $\pm$ 0.025) & 0.787 & 3006 & 8.0 $\pm$ 1.0 & 2.15 $\pm$ 0.06 & 17.4/11 \\
    & & & & & & \\
$r$ & -21.9 $ < M_{r} < $ -17.0 & -20.867 (1.596 $\pm$ 0.005) & 0.751 & 51080 & 7.9 $\pm$ 1.1 & 1.72 $\pm$ 0.05 & 12.1/16 \\
 -  & -22.4 $ < M_{r} < $ -21.9 & -22.140 (5.152 $\pm$ 0.009) & 0.775 & 6008 & 5.7 $\pm$ 1.0 & 2.15 $\pm$ 0.09 & 6.85/11 \\
 -  & -24.0 $ < M_{r} < $ -22.4 & -22.635 (8.127 $\pm$ 0.037) & 0.792 & 3004 & 7.9 $\pm$ 1.0 & 2.17 $\pm$ 0.06 & 14.8/11 \\
    & & & & & & \\
$i$ & -22.3 $ < M_{i} < $ -17.0 & -21.240 (2.033 $\pm$ 0.006) & 0.751 & 51078 & 7.8 $\pm$ 1.1 & 1.73 $\pm$ 0.05 & 12.5/16 \\
 -  & -22.7 $ < M_{i} < $ -22.3 & -22.478 (6.354 $\pm$ 0.011) & 0.776 & 6007 & 5.5 $\pm$ 1.0 & 2.17 $\pm$ 0.10 & 7.42/11 \\
 -  & -24.0 $ < M_{i} < $ -22.7 & -22.946 (9.778 $\pm$ 0.043) & 0.790 & 3005 & 7.3 $\pm$ 0.9 & 2.20 $\pm$ 0.06 & 11.8/11 \\
    & & & & & & \\
$z$ & -22.4 $ < M_{z} < $ -17.0 & -21.412 (2.360 $\pm$ 0.007) & 0.751 & 50697 & 8.0 $\pm$ 1.1 & 1.72 $\pm$ 0.05 & 12.2/16 \\
 -  & -22.8 $ < M_{z} < $ -22.4 & -22.563 (6.810 $\pm$ 0.011) & 0.772 & 5962 & 4.8 $\pm$ 1.1 & 2.16 $\pm$ 0.12 & 5.90/11 \\
 -  & -24.0 $ < M_{z} < $ -22.8 & -23.045 (10.61 $\pm$ 0.048) & 0.782 & 2983 & 6.5 $\pm$ 0.9 & 2.21 $\pm$ 0.07 & 10.6/11
\enddata
\end{deluxetable}



\end{document}
