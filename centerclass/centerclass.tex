%\documentclass{emulateapj}
\documentclass[12pt,preprint]{aastex}

\slugcomment{Last revision \today.}

\shortauthors{Sheldon et al.}
\shorttitle{Centering Tests}

% A comment block

%\newcommand{\comment}[1]{}

% For color
\newcommand{\mpname}[1]{#1_color.eps}
\newcommand{\clraitoff}{red}
\newcommand{\lumblack}{(black)}
\newcommand{\lumblue}{(blue)}
\newcommand{\lumred}{(red)}
\newcommand{\vdisred}{(red-dashed curve)}
\newcommand{\vdisblue}{(blue-solid curve)}

% For bw
%\newcommand{\mpname}[1]{#1.eps}
%\newcommand{\clraitoff}{}
%\newcommand{\lumblack}{}
%\newcommand{\lumblue}{}
%\newcommand{\lumred}{}
%\newcommand{\vdisred}{(dashed curve)}
%\newcommand{\vdisblue}{(solid curve)}

\newcommand{\umag}{$u$}
\newcommand{\gmag}{$g$}
\newcommand{\rmag}{$r$}
\newcommand{\imag}{$i$}
\newcommand{\zmag}{$z$}
\newcommand{\gmr}{$g-r$}



\newcommand{\gammat}{$\gamma_T$}
\newcommand{\gammacross}{$\gamma_\times$}
\newcommand{\deltasig}{$\Delta \Sigma$}
\newcommand{\deltaplus}{$\Delta \Sigma_+$}
\newcommand{\deltacross}{$\Delta \Sigma_\times$}
\newcommand{\deltarho}{$\Delta \rho$}
\newcommand{\movr}{$M(<r)$}
\newcommand{\sigmacrit}{$\Sigma_{crit}$}

\newcommand{\photoz}{photo-z}
\newcommand{\photozs}{photo-zs}

\newcommand{\tlum}{$L^{tot}$}
\newcommand{\tngal}{$N_{gal}^{tot}$}

\newcommand{\lstarlim}{$0.4 L_*$}
\newcommand{\lvir}{$L_{200}$}
\newcommand{\lvirtot}{$L^{tot}_{200}$}
\newcommand{\mvir}{$M_{200}$}
\newcommand{\nvir}{$N_{200}$}
\newcommand{\rvirgal}{$r_{200}^{gals}$}
\newcommand{\rvirmass}{$r_{200}^{mass}$}

\newcommand{\deltamtol}{$\Delta M/\Delta L$}
\newcommand{\deltam}{$\Delta M$}
\newcommand{\deltal}{$\Delta L$}

\newcommand{\deltamvir}{$\Delta M_{200}$}
\newcommand{\deltalvir}{$\Delta L_{200}$}

\newcommand{\mtolmax}{$(\Delta M/\Delta L)_{22\mathrm{Mpc}}$}
\newcommand{\mtolasym}{$(\Delta M/\Delta L)_{asym}$}
\newcommand{\mtolvir}{$(\Delta M/\Delta L)_{200}$}
\newcommand{\bmtol}{$b^2_{M/L}$}
\newcommand{\bmtolinv}{$b^{-2}_{M/L}$}

\newcommand{\ngal}{$N_{gal}$}
\newcommand{\maxbcg}{MaxBCG}
\newcommand{\numNgalBins}{12}
\newcommand{\numLumBins}{16}

\newcommand{\tngalAperture}{2$h^{-1}$ Mpc}

\newcommand{\photo}{\texttt{PHOTO}}
\newcommand{\astrop}{\texttt{ASTRO}}
\newcommand{\mt}{\texttt{MT}}
\newcommand{\spectro}{\texttt{SPECTRO}}
\newcommand{\spectroone}{\texttt{SPECTRO1d}}
\newcommand{\spectrotwo}{\texttt{SPECTRO2d}}
\newcommand{\target}{\texttt{TARGET}}

\newcommand{\lenszmax}{0.3}
\newcommand{\lenszmin}{0.05}
\newcommand{\zmean}{0.25}

\newcommand{\photoversion}{\texttt{v5\_4}}

%\def\eone{e$_1$}
%\def\etwo{e$_2$}
\newcommand{\etan}{e$_+$}
\newcommand{\erad}{e$_\times$}
\newcommand{\eclass}{\texttt{ECLASS}}
\newcommand{\eclasscut}{-0.06}
\newcommand{\gmrcut}{0.7}

\newcommand{\hrs}{$^{\mathrm h}$}
\newcommand{\minutes}{$^{\mathrm m}$}

\newcommand{\ugriz}{$u, g, r, i, z$}
\newcommand{\polarization}{polarization}

\newcommand{\wgm}{$w_{gm}$}
\newcommand{\wgg}{$w_{gg}^p$}
\newcommand{\wmm}{$w_{mm}$}
\newcommand{\xigg}{$\xi_{gg}$}
\newcommand{\ximm}{$\xi_{mm}$}
\newcommand{\xigm}{$\xi_{gm}$}

\newcommand{\numspec}{127,001}
\newcommand{\numspecvlim}{10,277}
\newcommand{\numrand}{1,270,010}
\newcommand{\numspectot}{278,192}
\newcommand{\numvdis}{49,024}
%\newcommand{\numsource}{10,259,949}
% hirata: 
\newcommand{\nummask}{1,815,043}
\newcommand{\numTenMpc}{132,473}
\newcommand{\numThirtyMpc}{101,221}
\newcommand{\numsource}{27,912,891}

\newcommand{\numpairsTenMpc}{2,670,898,177}
\newcommand{\altnumpairsTenMpc}{2.7 billion}
\newcommand{\numpairsThirtyMpc}{14,818,082,122}
\newcommand{\altnumpairsThirtyMpc}{14.8 billion}



\newcommand{\xirmax}{$\xi_{gm}(R_{max})$}



\newcommand{\cbcg}{$c_{bcg}$}
\newcommand{\ckde}{$c_{kde}$}
\newcommand{\clkde}{$c_{lkde}$}

\newcommand{\cclass}{\texttt{cclass}}

\begin{document}


\title{Testing Center Classes}



\author{
Erin S. Sheldon,\altaffilmark{1}
}

\altaffiltext{1}{Center for Cosmology and Particle Physics, Department of
Physics, New York University, 4 Washington Place, New York, NY 10003.}


\section{Introduction} \label{sec:intro}

Ben's original catalog uses the BCG as the center.  Using the same clusters new
centers were calculated taking the peak in a Gaussian smoothed density map,
also known as a ``kernel density estimate''.  The scale length of the Gaussian
is roughly 200 kpc, although it varies with richness.  This peak was found with
luminosity weighting and without.  We will call these three centers \cbcg,
\ckde, and \clkde.  We will call the distance between centers $d$, e.g.  the
distance between \cbcg\ and \ckde\ is denoted $d(bcg,kde)$.

Eli looked at the distribution of offsets from \cbcg\ and fit a Rayleigh 
distribution to each.  He found a sigma for of 40 kpc for $d(bcg,lkde)$
and for $d(bcg,kde)$ it is 50-60 kpc.  

Based on the new peaks the following center classes were defined.  In all cases
the clusters have $N_{200} \geq 5$.  

\begin{itemize}
    \item \cclass\ 0: ``Good'' centers.  $d(bcg,kde) < 3 \sigma$ and $d(bcg,lkde) < 3 \sigma$.
    \item \cclass\ 1: ``Bad'' centers.  $d(bcg,kde) > 8 \sigma$ and $d(bcg,lkde) > 8 \sigma$.
    \item \cclass\ 2: $d(bcg,kde) > 8 \sigma$ and $d(bcg,lkde) < 3 \sigma$.
    \item \cclass\ 3: $d(bcg,kde) < 3 \sigma$ and $d(bcg,lkde) > 8 \sigma$.
    \item \cclass\ 4: $3 < d(bcg,kde)/\sigma < 8$ or $ 3 < d(bcg,lkde)/\sigma < 8$.

    \item \cclass\ -1: Unclassified.  no \ckde\ or \clkde\ calculated, $z >
    0.3$ or $z < 0.1$.

\end{itemize}

\newpage
Here are some statistics for the samples:
\begin{verbatim}
  class  nlens      N200 err(N200)      L200 err(L200)      Lbcg err(Lbcg)
     -1   3183     10.51      0.24     20.01      0.45      5.67      0.08
      0  23441      8.82      0.06     16.10      0.11      4.97      0.02
      1   2037     10.88      0.25     18.30      0.46      3.57      0.05
      2   5739      9.06      0.12     16.64      0.23      5.09      0.05
      3     86     14.80      1.60     26.19      3.09      3.91      0.26
      4  14805      8.21      0.06     14.63      0.12      4.45      0.03
\end{verbatim}

Note the difference in mean \nvir\ between the various samples.   First let us
compare \cclass\ 0 to \cclass\ 1, ``good'' centers versus ``bad'' centers.  The
mean \nvir\ is larger for \cclass\ 1 than for \cclass\ 0, as is the \lvir.  The 
BCG luminosity is lower however.  This same pattern holds when comparing
\cclass\ 0 to \cclass\ 3.  But \cclass\ 2 is very similar to \cclass\ 0, 
as is \cclass\ 4.

This comparison is shown in Figure \ref{fig:compstat} as a ratio between
each class and \cclass\ 0.

\begin{figure}[h]
\plotone{plots/maxbcg_sample21-22_centerclass6_corr_stats.eps}

\caption{Comparison between some statistics of the \cclass\ 0 clusters
(``good'' centers) and other classifications, expressed as a ratio
compared to \cclass\ 0.  The trends in \nvir\ and \lvir\ tend
to be reversed in BCG luminosity.} \label{fig:compstat}

\end{figure}

\begin{figure}[h]
\plotone{plots/maxbcg_sample21-22_centerclass6_corr_color_compare.eps}

\caption{Comparison between \cclass\ 0 clusters (``good'' centers) and other
classifications.} \label{fig:comp}

\end{figure}






\section{Comparison of Lensing Signals for Center Classes}

Figures \ref{fig:comp} and \ref{fig:comprescale} show comparisons of the
lensing signal between the different classes.  For Figure \ref{fig:comp}, the
differences in mean richness between each \cclass\ make the comparison difficult. 

In order to remove the confusion from different richnesses, and focus more on
shape,  I rescaled the \cclass\ 0 results so that the mean of the
points beyond 2 Mpc equals that of the comparison class. These results are shown
in figure \ref{fig:comprescale}.  This is especially helpful for comparing
\cclass\ 0 and \cclass\ 1.

In each case the \cclass\ 0 objects have a steeper slope and higher amplitude
relative to the large scales in comparison to the other classes. 

Note on small scales the comparison is complicated by the presence of the
BCG.



With this current analysis it is somewhat difficult to interpret the results in
terms of centering.  First of all, there is a BCG at the center in all cases
and this dominates the shape of the profile on small scales, independent of the
centering.  Furthermore, the objects in each class do not have similar
richnesses, and the BCGs are not of comparable luminosity.

Despite this we can say something:  the one-halo term is generally stronger for
the \cclass\ 0 clusters.  For example, \cclass\ 0 has a stronger one-halo
term than \cclass\ 1 even though \cclass\ 1 objects have a higher mean
richness.  In the next section we will define samples that are better
matches.


\begin{figure}[h]
\plotone{plots/maxbcg_sample21-22_centerclass6_corr_color_compare_rescaled.eps}

\caption{Comparison between a rescaled \cclass\ 0 (``good'' centers) and other
classifications.  In each panel the \cclass\ 0 curve has been rescaled so that
the mean of the points greater than 2 Mpc is equal to that of the comparison
data.  This helps to account for differences in the mean richness between the
different classes.  After the rescaling the differences at intermediate radii,
which should be in the halo and thus more sensitive to centering, are more
apparent.}  \label{fig:comprescale}

\end{figure}



\section{Comparison of Luminosity-matched Samples}

I removed objects from the \cclass\ 0 such that the $i$-band luminosity
histogram was proportional to that of the \cclass\ 1 sample, with the further
restriction that $0.45 < \mathrm{log}(L/L_\odot) < 2.0$ to prevent noise in the
histogram from being important.  Here are statistics for these new samples:
\begin{verbatim}
  class  nlens      N200 err(N200)      L200 err(L200)      Lbcg err(Lbcg)
      0  12887      9.30      0.07     17.24      0.13      5.25      0.03
      1   2019     10.18      0.16     17.01      0.29      3.50      0.04
\end{verbatim}

Figure \ref{fig:compmatch} shows the comparison between these samples.  
There remains a clear change in shape after matching luminosities, with
a stronger one-halo term, although the effect is somewhat suppressed 
compared to the unmatched tests.  Note a ``rescaled'' test, matching
large scale amplitudes, does not change the result significantly as the
large scale amplitudes are already very similar.

\begin{figure}[h]
\epsscale{0.6}
\plotone{plots/maxbcg_sample21-22_centerclass_alt2_corr_color_compare.eps}

\caption{Comparison between \cclass\ 0 clusters (``good'' centers) and \cclass\
1 (``bad'').  For these measurements the samples have been matched to have the
same distribution of $i$-band luminosity.} \label{fig:compmatch}
\epsscale{1}

\end{figure}


\end{document}

