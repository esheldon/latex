\documentclass{beamer}

\usepackage{beamerthemesplit}
\usepackage{verbatim}

\usepackage{xcolor}

\definecolor{gold}{rgb}{1.,0.84,0.}
\definecolor{brightred}{rgb}{1.,0.4,0.4}
\definecolor{mygray}{RGB}{200,200,200}
\definecolor{lightsteelblue}{RGB}{176,196,222}
\definecolor{lightskyblue}{RGB}{135,206,250}
\definecolor{cadetblue}{RGB}{95,158,160}

\usetheme{default}
\usecolortheme{mule}

\usefonttheme{serif}

%\DeclareGraphicsExtensions{.pdf,.png,.jpg}

\newcommand{\snT}{$(S/N)_{\textrm{size}}$}
%\newcommand{\snT}{$\left( \frac{S}{N}\right)_{\textrm{size}}$}
\newcommand{\snflux}{$(S/N)_{\textrm{flux}}$}
%\newcommand{\snflux}{$\left( \frac{S}{N}\right)_{\textrm{flux}}$}

\newcommand{\lensfit}{\texttt{LENSFIT}}
\newcommand{\numba}{\texttt{Numba}}
\newcommand{\python}{\texttt{Python}}
\newcommand{\ngmix}{\texttt{ngmix}}
\newcommand{\shear}{{\bf g}}
\newcommand{\redmapper}{redMaPPer}

\newcommand{\prelim}{{\bf{\it Preliminary}}}



\title{Update on Metacalibration for Weak Lensing Shear Measurement}
\author{Erin Sheldon}
\institute{Brookhaven National Laboratory}

% http://texblog.net/latex-archive/plaintex/beamer-footline-frame-number/
% to add the page (frame ) number and not screw up the bottom line
% works for split themes?
\expandafter\def\expandafter\insertshorttitle\expandafter{%
      \insertshorttitle\hfill%
        \insertframenumber\,/\,\inserttotalframenumber}

% suppress navigation bar
\beamertemplatenavigationsymbolsempty
\setbeamertemplate{footline}{}

\begin{document}

\frame{\titlepage}


\setbeamertemplate{background canvas}[vertical shading][bottom=mgray,top=mblack]

\frame
{
    \frametitle{Outline}

    \setbeamerfont*{itemize/enumerate body}{size=\Large}
    \setbeamerfont*{itemize/enumerate subbody}{parent=itemize/enumerate body}
    \setbeamerfont*{itemize/enumerate subsubbody}{parent=itemize/enumerate body}
 
    \begin{itemize}

        %\item The Primary Goal is to Study Dark Energy
        \item Metacalibration

        \item Correlated Noise

        \item Correction for Correlated Noise

        \item Performance on Realistic Simulations

    \end{itemize}

}

\frame
{
    \frametitle{Shear Accuracy Requirements}

    \setbeamerfont*{itemize/enumerate body}{size=\Large}
    \setbeamerfont*{itemize/enumerate subbody}{parent=itemize/enumerate body}
    \setbeamerfont*{itemize/enumerate subsubbody}{parent=itemize/enumerate body}
 
    \begin{itemize}

        \item In order to measure the Dark Energy equation of state
            to the desired accuracy for DES/LSST, we must measure
            shear with exquisite accuracy.

        \item Shear calibration errors
            \begin{itemize}
            
                \item {\color{lightskyblue} DES}:~ $\Delta \gamma/\gamma \lesssim 0.004$
                \item {\color{brightred} LSST}: $\Delta \gamma/\gamma \lesssim 0.001$

            \end{itemize}


    \end{itemize}

}



\frame
{
    \frametitle{Metacalibration Idea from Eric Huff}

    \setbeamerfont*{itemize/enumerate body}{size=\normalsize}
    \setbeamerfont*{itemize/enumerate subbody}{parent=itemize/enumerate body}
    \setbeamerfont*{itemize/enumerate subsubbody}{parent=itemize/enumerate body}
 
    \begin{itemize}

        \item Say we have a biased shear estimator {\color{gold} $E$}.  Then we can write
            {\color{gold}
                \begin{eqnarray}
                    E & = & E(\gamma=0) + \frac{\partial E}{\partial \gamma} \gamma + ... \nonumber \\
                      & \sim &  \frac{\partial E}{\partial \gamma} \gamma \equiv R \gamma \nonumber 
                \end{eqnarray}
            } 
        \item Use image manipulation to estimate the derivative of the
            estimator with respect to shear
            {\color{gold}
                \begin{equation}
                    R = \frac{E(+\Delta\gamma) - E(-\Delta\gamma)}{2 \Delta \gamma} \nonumber 
                \end{equation}
            }
            \begin{itemize}
                \item Deconvolve the PSF
                \item Shear the image by a small amount
                \item Reconvolve by the PSF.  Use a slightly larger PSF to suppress
                    the noise amplification
            \end{itemize}


    \end{itemize}

}

\frame
{
    \frametitle{Metacalibration Idea from Eric Huff}

    \setbeamerfont*{itemize/enumerate body}{size=\Large}
    \setbeamerfont*{itemize/enumerate subbody}{parent=itemize/enumerate body}
    \setbeamerfont*{itemize/enumerate subsubbody}{parent=itemize/enumerate body}
 
    \begin{itemize}
        
        \item Corrects for modeling biases

        \item Corrects for {\em ordinary} noise-related biases

        \item Works well at high shear.

    \end{itemize}

}



\frame
{
    \frametitle{Correlated Noise}

    \setbeamerfont*{itemize/enumerate body}{size=\large}
    \setbeamerfont*{itemize/enumerate subbody}{parent=itemize/enumerate body}
    \setbeamerfont*{itemize/enumerate subsubbody}{parent=itemize/enumerate body}
 
    \begin{itemize}

        \item These convolutions and shears result in {\em {\color{gold} correlated noise}}
            
        \begin{itemize}
            \item After convolution, fluctuations due to noise are no longer
                independent between pixels

            \item Shearing involves interpolation, so in a similar way fluctuations
                due to noise are no longer independent
        \end{itemize}

        \item Can result in bias of order $5-10$\% for very faint galaxies.


    \end{itemize}

}

\frame
{
    \frametitle{Correlated Noise Example}

    \begin{figure}
        \includegraphics[width=0.75\textwidth]{metacal_noise_images_neg.png}
    \end{figure}
    Figure: Eric Huff (OSU)
}

\frame
{
    \frametitle{Correlated Noise}


    \begin{itemize}

        \item Cancels from mean estimator
            {\color{lightskyblue}
                \begin{equation}
                    E = \frac{E(+\Delta \gamma) + E(-\Delta\gamma)}{2} \nonumber
                \end{equation}
            }

        \item Does not cancel from $R$
            {\color{gold}
                \begin{equation}
                    R = \frac{E(+\Delta\gamma) - E(-\Delta\gamma)}{2 \Delta \gamma} \nonumber 
                \end{equation}
            }

    \end{itemize}

}




\frame
{
    \frametitle{Old Correction for Correlated Noise}

    \setbeamerfont*{itemize/enumerate body}{size=\large}
    \setbeamerfont*{itemize/enumerate subbody}{parent=itemize/enumerate body}
    \setbeamerfont*{itemize/enumerate subsubbody}{parent=itemize/enumerate body}
 
    \begin{itemize}

        \item Originally I was using deeper data to avoid the
            correlated noise: degrade to the
            noise level of the shallow data, but adding noise {\em after}
            performing metacal convolutions/shear.

        \item This has significant drawbacks
            \begin{itemize}
                \item Deep data is {\color{brightred} expensive} to acquire

                \item Great care must be taken that the deep data is well matched
                    to the shallow data
            \end{itemize}

    \end{itemize}

}


\frame
{
    \frametitle{New Correction for Correlated Noise}

    \setbeamerfont*{itemize/enumerate body}{size=\large}
    \setbeamerfont*{itemize/enumerate subbody}{parent=itemize/enumerate body}
    \setbeamerfont*{itemize/enumerate subsubbody}{parent=itemize/enumerate body}
 
    \begin{itemize}

        \item Corrections can be derived from the shallow data itself

        \item The best idea we have:
            \begin{itemize}

                \item The bias due to correlated noise should scale with the
                    {\color{gold} correlation function} of the noise.
                    This is intuitive, but it also has been derived in general
                    by (Hirata, private communication)

                \item Bias thus scales with the {\color{gold} noise} amplitude squared

                \item {\em Add a little noise} and look for this scaling, remove trend.

            \end{itemize}
    \end{itemize}

}



\frame
{
    \frametitle{Detrending Correction for Correlated Noise}

    {\color{gold} $\Delta R_{\mathrm{noise}} = A \Delta n^2$ }

    \begin{figure}
        \includegraphics[width=0.6\textwidth]{run-bd13mcal-dt01-Rnoise-detrend-neg.png}
    \end{figure}

    Note offset is not zero.  Use 
    {\color{lightskyblue} $R_{\mathrm{noise}} = A n^2 - \mathrm{offset}$}
}

\frame
{
    \frametitle{Performance on Simulations}

    \setbeamerfont*{itemize/enumerate body}{size=\large}
    \setbeamerfont*{itemize/enumerate subbody}{parent=itemize/enumerate body}
    \setbeamerfont*{itemize/enumerate subsubbody}{parent=itemize/enumerate body}
 
    \begin{itemize}
        \item Simulations with complex galaxies:
            \begin{itemize}
                \item bulge+disk
                \item Large offsets between bulge and disk centers.
            \end{itemize}

        \item I fit a simple gaussian, which normally results in a large
            ``model bias'', of order 10\%.
            
         \item Signal-to-noise ratio distribution matched to real data, with
             lower bound $\gtrsim$10.  Induces ordinary noise bias of order
             10\%

    \end{itemize}

}

\frame
{
    \frametitle{Performance on Simulations}

    \setbeamerfont*{itemize/enumerate body}{size=\large}
    \setbeamerfont*{itemize/enumerate subbody}{parent=itemize/enumerate body}
    \setbeamerfont*{itemize/enumerate subsubbody}{parent=itemize/enumerate body}
 
    \begin{itemize}
            
            
         \item Model the bias as a multiplicative and an additive part
        {\color{lightskyblue} 
            \begin{equation}
                \gamma = (1 + m ) \times \gamma_{true} + c \nonumber
            \end{equation}
        }


         \item With correlated noise corrections, the biases are reduced
             by at least two orders of magnitude

        {\color{gold} 
            \begin{eqnarray}
                m & = & (1.5 \pm 2.0) \times 10^{-3} \nonumber \\
                c & = & (-3.4 \pm 7.0) \times 10^{-5} \nonumber
            \end{eqnarray}
        }
         \item Running now on a larger sim with real galaixes from COSMOS, DES PSF,
            will reach a precision of $\sim 0.7 \times 10^{-3}$

    \end{itemize}

}



\frame
{
    \frametitle{Summary}

    \setbeamerfont*{itemize/enumerate body}{size=\large}
    \setbeamerfont*{itemize/enumerate subbody}{parent=itemize/enumerate body}
    \setbeamerfont*{itemize/enumerate subsubbody}{parent=itemize/enumerate body}
 
    \begin{itemize}
        \item Metacalibration is a new idea for shear recovery from
            Eric Huff

        \item Promising new correlated noise corrections that work {\color{gold} {\em without the need
            for expensive deep data} }

        \item On tests so far, the bias is within statistical error.  More simulations are running.
        
    \end{itemize}

}

\end{document}
