\begin{deluxetable}{lcccc}
\tabletypesize{\small}
\tablecaption{Projected Computing Purchases\label{table:computing}}
\tablewidth{0pt}
\tablehead{
	\multicolumn{1}{l}{Fiscal Year} &
	\colhead{Disk Storage}       & 
	\colhead{\$ for Storage}    & 
	\colhead{Compute Servers}   & 
	\colhead{\$ for CPU} \\
	&
	[TB] &
	&
	2010 Equivalent &
}
\startdata
2012 & 34 & 9600 & 32 & 63000 \\
2013 & 44 & 9600 & 40 & 63000 \\
2014 & 55 & 9600 & 50 & 63000 \\
\hline
\relax\\[-1.7ex]
Total in 3 years & 133 & 28800 & 122 & 189000 \\\\[-2.7ex]
\enddata

\tablecomments{The number of compute nodes purchased is based on
the assumption that each node (26kSI2k, 104 HEP-SPEC 2006) would stay at the
performance level of a node purchased in 2010. As the performance per node will
increase over time the actual number of compute nodes after 3 years will be
significantly smaller (probably O(70)), providing a combined performance of
O(122) 2010 equivalent nodes. Prices include 40\% bulk discounts from
purchasing through the RHIC ATLAS Computing Facility at BNL. {\bf Power,
cooling and maintence will be provided at no extra cost.} Although we list
storage purchases per year, storage most likely be bought in two or
three larger chunks.  }

\end{deluxetable}

