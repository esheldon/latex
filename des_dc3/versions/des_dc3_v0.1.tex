\documentclass[12pt]{article}

\newcommand{\psforderval}{6}
\newcommand{\ncoeff}{\texttt{Ncoeff}}
\newcommand{\ncoeffval}{28}

\title{Plan and Progress for DES Lensing Code and Metadata} 
\author{Erin Sheldon}

\begin{document}

\section*{Plan and Progress for DES Lensing Code and Metadata} 

The codes required for generating unbiased galaxy shapes:

\begin{enumerate}
    \item A code to read DES images and object lists for a single-epoch (SE).
        It is planned to run on the fits outputs rather than reading from the
        database (DB).
    \item A code for finding PSF stars in a SE.
    \item A code for decomposing a PSF star on a SE into a set of shapelet 
        coefficients, plus a wrapper to perform this on a list of such stars.
        Outputs are sent to a local file.
    \item A code to performe shapelet decomposition for all objects in a SE 
        without ``deconvolving''.  This uses the same core code as for PSF
        determination, but with different configuration parameters.  Also
        fewer coefficients will be kept.  Outputs are sent to a local file.
    \item A code to ingest the output files into the DB.  There will be
        separate tables for PSF stars data and all objects data.
    \item A code to interpolate the PSF to any position in a SE.  This will be
        a straightforward interpolation using only the information in the SE.
    \item A code to interpolate the PSF using the Jarvis \& Jain PCA scheme.
    \item A code to retrieve the image data, catalogs, and PSF interpolation
        for all SEs covering a given region of sky.
    \item A code to perform galaxy shapelet decomposition ``deconvolved'' from
        the PSF.  This is not actually a deconvolution, but a forward modeling
        where the shapelet basis is convolved with the estimated PSF shapelets
        decomposition at the position of the object before the fit is 
        performed.  
    \item An extended code will fit ``deconvolved'' galaxy shapelets across 
        all available exposures for a given object simultaneously.
\end{enumerate}

\section{Shapelets for PSF Stars for a Single-Epoch} \label{sec:sepsf}


This code exists and was delivered to the DES-DM group 2006-12-09.  It includes
the code for reading the FITS SE images and catalog files\footnote{The fitsio
package is currently a part of this package, but could be linked in from
pre-build binaries elsewhere}.  This original code is in a SVN repository at
NYU tagged as \texttt{v1.0} and a tar ball can be downloaded here:
\begin{verbatim}
http://sdss.physics.nyu.edu/esheldon/des/code/des_shapelets-2006-12-09.tar.gz
\end{verbatim}

This code reads the SE data and performs a ``first pass'', decomposing all
objects into low order (n=2) shapelets to determine a robust size. Stars are
separated from galaxies, cosmic rays, and saturated objects, and a suitable set
of high significance stars are chosen for a decomposition to higher order
in the second pass.

Two output files are currently created, one for the PSF stars and one for all
objects.  These are currently comma separated value (CSV) ASCII files.  The
table structure for ``all'' and PSF objects are described in tables
\ref{tab:allstruct} and \ref{tab:psfstruct} respectively.  The datatypes listed
in these tables are the recommended storage in the database. Note, flags
accrued in the first pass are kept through the second pass.

\begin{table}[p]
    \small
    \begin{tabular}{cccl}
    column & name & datatype & description\\
    \hline
    1 & id       & 64-bit integer &  Unique DES id number \\
    2 & mag\_auto & 32-bit float   & SExtractor mag\_auto \\
    3 & flags1   & 32-bit integer & Information flags about processing \\
    4 & sigma    & 32-bit float   & Equivalent gaussian width output by 
                                    the shapelet code \\
    \hline
    \end{tabular}
    \begin{center}
        \caption{Table structure for ``all'' objects \label{tab:allstruct}}
    \end{center}
    \normalsize
\end{table}
\begin{table}[p]
    \small
    \begin{tabular}{cccl}
    column & name & datatype & description\\
    \hline
    1 & id       & 64-bit integer &  Unique DES id number \\
    2 & mag\_auto & 32-bit float   & SExtractor mag\_auto \\
    3 & flags1   & 32-bit integer & Information flags about processing \\
    4 & sigma    & 32-bit float   & Equivalent gaussian width output by 
                                    the shapelet code \\
    5-(Ncoeff+4) & coeff & 32-bit float & Ncoeff shapelet coefficients \\
    \hline
    \end{tabular}
    \begin{center}
        \caption{Table structure for PSF stars \label{tab:psfstruct}}
    \end{center}
    \normalsize
\end{table}



The variable \texttt{Ncoeff} in the tables is the total number of coefficients
in the shapelet expansion.  This is related to the order of the expansion $n$ 
by the following equation
\begin{equation}
    \texttt{Ncoeff} = n*(n+3)/2 + 1
\end{equation}
We are currently using $n = $\psforderval\ for the PSF decomposition, giving
\ncoeff=\ncoeffval.

{\bf Question: } Should the ``all'' outputs be ingested as a separate table
from the main object table?  It might be better to merge the data before
ingestion, since table joins are really slow.  On the other hand, separate
tables would allow flexibility, say if we wanted to re-run the lensing
code separately.

{\bf Status:} v1.0 tested and submitted 2006-12-09.  Ingestion code, database
testing, and testing in real world pipleline situation needed.

\section{Shapelets for All Objects, Not Accounting for the PSF}
\label{sec:seall}

This is essentially already being performed as the first pass described above.
Currently this pass is only $n=2$ because all that was needed was a robust size
measurement.  We could simply output these 6 coefficients, or increase the
order a bit which will result in a somewhat lower convergence rate.  Either way
the time spent on this task will be domnated by the test phase rather than
writing new code.  The table structure for such outputs would be the same as
for \ref{tab:psfstruct} but with a different value for \ncoeff.

{\bf Status:} Not finished, but all the pieces are in place.  Need to decide
what order we need.

{\bf Deadline:}  November 1.

\section{PSF Interpolation in a Single-Epoch} \label{sec:seinterp}

Mike already has code to do this, we just need to implement it and decide
on the spatial frequencies we can constrain.

{\bf Status:} Code exists that needs to be adapted.

{\bf Deadline:} March?

\section{Pre-PSF Galaxy Shapes in Single-Epoch Data} \label{sec:deconv}

The idea is to extract the true galaxy shape by fitting a model that is already
convolved with the PSF.  This convolution is simple in shapelet space. What is
needed is reliable PSF interpolation from \S \ref{sec:seinterp}.  A code
exists that has been tested on simulations and Mike is working on it.

{\bf Status:}  A code exists that needs to be adapted to our needs.

{\bf Deadline:} March?

\section{PSF Interpolation in Multiple-Epoch Data} \label{sec:meinterp}

{\bf Status:} Unfinished.  An implementation for the 75 square degree survey
exists which needs to be adapted to the DES.  We need to address the data
management issues and explore the parameter space.

{\bf Deadline:} ??

\section{Pre-PSF Galaxy Shapes from Multiple Exposures} \label{sec:mdeconv}

Fit for pre-PSF galaxy shape using all available exposures for each object.
Essentially this is combining the $\chi^2$ over all exposures, fitting the same
model convolved with the PSF for each exposure.  The is conceptual only, and
there are some details to consider such as what pixels to use from each
exposure.  The biggest hurtle in pracice will probably be the data management
problem of randomly extracting all SE images and catalogs relevant for a given 
region of sky.

{\bf Status:} Conceptual only.
{\bf Deadline:} 2008?


\end{document}
