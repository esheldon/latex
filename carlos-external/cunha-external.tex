%\documentclass[11pt,preprint]{emulateapj}
%\documentclass[oneside,letterpaper,12pt]{article}
%\documentclass[letterpaper,12pt]{article}
\documentclass[11pt]{article}

\usepackage{graphicx,amssymb}
\usepackage{graphicx,natbib}
\usepackage{url}
\usepackage{color}
\usepackage{setspace}
\doublespace
 
\citestyle{aa}

\renewcommand{\baselinestretch}{2}
\setlength{\topmargin}{-0.6in}
\setlength{\oddsidemargin}{0.pt}
\setlength{\evensidemargin}{0.0pt}
\setlength{\textheight}{9.1in}
\setlength{\textwidth}{6.52in}

\newcommand{\zphot}{$z_{phot}$}
\newcommand{\zspec}{$z_{spec}$}
\newcommand{\apj}{ApJ}
\newcommand{\aj}{AJ}
\newcommand{\aap}{A\&A}
\newcommand{\mnras}{MNRAS}
\newcommand{\pasp}{PASP}
\newcommand{\prd}{Phys. Rev. D}


\begin{document}   
\begin{singlespace}      
%\maketitle
%\title{Statement of Research Interests}
\begin{center}
\textbf{\large{External Collaborator Request for Carlos Cunha}}\\
\end{center}

%\section*{Introduction}

We request External Collabortor status for Carlos Cunha, University
of Michigan. Cunha will participate in a project to estimate redshift
distributions for galaxies in the SDSS III imaging. This project
has already been announced as ``SDSS-III Project 51: A Photometric
Redshift Catalog for SDSS DR8''. Cunha will participate in this {\bf one
project only}. 

The purpose of this request is to include Cunha in the author list,
not to gain him access to proprietary data. Cunha will not handle
any proprietary data for this project. He has assembled a number of
spectroscopic training sets, which are all public. He has provided 
code, which Erin Sheldon has modified and is running on the imaging
data, and he has and will contribute text to the paper. The new BOSS
redshifts may be included in the training set at some point, but in that
case the data will be handled by Erin Sheldon not Cunha.

While there are other projects involving photometric redshifts in
the SDSS III, the techniques Cunha has developed are unique and
complimentary. The ``weighting'' method can robustly derive redshift
distributions for diverse galaxy populations \citep{lim08,cun09a},
whereas other projects focus on particular types of galaxies or
estimation of a single ``best redshift''. The generation of full
redshift distributions is particularly useful for studies such as
gravitational lensing that inherently involve expectation values of
physical quantities over a redshift distribution (note Cunha will not
work on those projects).

Cunha already has experience working with SDSS data (see below). While
this project could in principle be performed after DR8 is released,
generating these redshift distributions on a shorter timescale will
facilitate other studies that can begin well before the data release,
such as galaxy cluster lensing or interpretation of correlation
functions using photometrically selected galaxies (although note Cunha
will not work on said projects). 

Furthermore, we plan to release these catalogs to the public
through the SDSS website along with DR8. This would be a catalog
release, not part of the database. This follows the form of
previous photoz releases as ``value added catalogs''\citep{dr7vac}.
\newline
\newline
\textbf{Previous Experience}
\newline

Cunha has extensive experience in photometric redshift estimates, and
has applied these techniques to SDSS data. Cunha and collaborators
have developed, tested, and improved a number of photo-z estimators
\citep[e.g.][]{oya06}. In \cite{oya08b} he and collaborators tested
error estimates derived from both template-fitting and training-set
techniques. He developed a new photo-z error estimation technique, the
nearest neighbor error estimate (NNE), which uses the error distribution
of the nearest spectroscopic training-set objects (in magnitude space)
to associate a photo-z error to objects for which no spectroscopic
redshift is available (hereafter the photometric sample). He found that
whenever a reasonable training set is available, the NNE is superior
to other photo-z error estimators. As a concrete application of the
above photo-z techniques, he produced a photo-z catalog for 77 million
galaxies in the SDSS \citep{oya08a}. He and collaborators applied the
neural network technique to galaxy magnitudes, colors, and concentration
parameters.

The estimation of photo-z's relies on using multi-band photometry to
determine the location of galaxy spectral features in wavelength space.
The accuracy and reliability of photo-z estimates therefore depend on
the number, wavelength range, and relative positions of the filters,
as well as the signal-to-noise achieved in each. Cunha developed a
Markov Chain Monte Carlo code to optimize the choice and exposure times
of filters for photo-z estimation and applied it to study the filter
choices for DES. The results of this study have been incorporated into
the specifications of the DES filters \citep{dep08}.

More recently, Cunha studied the estimation of galaxy redshift
distributions from photometric data. Applications such as weak lensing
and baryon acoustic oscillations only require estimates of the galaxy
redshift distribution, $N(z)$, not individual photo-z estimates. Since
photo-z estimates can be biased it is important to develop techniques
for optimal estimation of $N(z)$ on its own right, perhaps independently
of photo-z's. He and collaborators developed an estimator for
$N(z)$ which involves weighting galaxies in a spectroscopic training set
so that the weighted distributions of colors and magnitudes match those
of the photometric sample. This technique bypasses photo-z estimation to
directly and more accurately estimate $N(z)$ \citep{lim08,cun09a}.

Finally, Cunha has found the weighting technique can be extended
to yield estimates of the redshift probability distributions, $p(z)$'s,
for each galaxy in the photometric sample \citep{cun09a}. Many of the
biases seen in photo-z estimates arise because the photo-z estimator
selects a single number to represent the full probability distribution.
Working with the full $p(z)$ for each object one can get a much
better estimate of the redshift distribution of the entire sample. 
\cite{man08} demonstrated that the use of this $p(z)$ estimate
yields a smaller lensing calibration bias than using individual galaxy
photo-z estimates. Cunha made public a catalog of p(z)'s for about 88
million galaxies of the SDSS Data Release 7 \citep{cun09a}.
\newline
\newline
\noindent
Erin Sheldon
\newline
Rachel Mandelbaum
\newline
Carlos Cunha

\bibliographystyle{apj}
\bibliography{past2010}
\end{singlespace}
\end{document}
