\documentclass[preprint]{aastex}

\begin{document}

\title{maxBCG}
\author{Erin Sheldon, based on Jim Annis' and Ben Koester's algorithms}

\section{Bayes Theorem}

We assume that clusters of galaxies have a special galaxy with a known
distribution of colors as a function of redshift, and ``member'' galaxies on
the ES0 ridgeline, nearby in space.  We will denote the special galaxy as a
Brightest Cluster Galaxy (BCG), although the utility of demanding it is the
brightest may be a matter of debate. 

We go to each galaxy in the survey and, for a given redshift and metric
aperture, count the neighboring galaxies within the aperture and ridgeline. We
then wish to find the most likely redshift and member population.  Using Bayes'
theorem:

\begin{equation} \label{eq:posterior}
P(z, n_{gals} | m_{BCG}, N_{ap}) = \frac{ P(m_{BCG}, N_{ap} | z, n_{gals}) P(z, n_{gals}) }{P(m_{BCG}, N_{ap})},
\end{equation}
where $n_{gals}$ is the number of member galaxies for the cluster, $z$ is the
cluster redshift, $m_{BCG}$ is the measured magnitudes for the BCG candidate,
and $N_{ap}$ is the measured number of galaxies in the metric aperture, which
is a function redshift in angular space.

We will take flat priors on $z$ and $n_{gals}$, so all we need to do is
maximize the likelihood function.  We will assume that the BCG likelihood
separates from the member likelihood
\begin{equation} \label{eq:likelihood}
P(m_{BCG}, N_{ap} | z, n_{gals}) = P(m_{BCG}|z)P(N_{ap} | z, n_{gals})
\end{equation}
The first term is measured from known BCG's by examining their magnitudes
as a function of redshift.  The second term involves the number of galaxies
in the aperture
\begin{equation} \label{eq:nap_like}
N_{ap} = n_{gals} + N_{ap}^{random},
\end{equation}
where $n_{gals}$ is the quantity of interest, and $N_{ap}^{random}$ is the
contribution from some uniform component of galaxies not associated with the
cluster.  This is measureable using random points for a given redshift, and
should be a poisson random variable with distribution $P(N_{ap}^{random})$ and
mean $\langle N_{ap}^{random} \rangle$.  The most likely $n_{gals}$ is given by
$N_{ap} - \langle N_{ap}^{random} \rangle$.  Of course, the full
two-dimensional likelihood should be explored.

This differs from Jim's likelihood function:
\begin{equation}
P(m_{BCG}, N_{ap} | z, n_{gals}) = N_{ap}(z)P(m_{BCG}|z) = 
        n_{gals}(z)P(m_{BCG}|z)
\end{equation}
I don't think that maximizing this function gives the most likely $n_{gals}$
and $z$ in general.  For high enough $n_{gals}$ and high $z$, however, it may
in fact recover the best values because the random component is negligible.  It
may also bias the redshift higher since larger apertures will increase
$n_{gals}$, but comparisons with spectroscopic redshifts do not reveal any such
bias.

\section{Discussion}

Maximizing the above likelihood gives the best $z$ and $n_{gals}$ for the 
BCG candidate.  This does not, however, find the best cluster candidates.
We need some kind of ranking system for this.  Probably some combination
of $n_{gals}$, the likelihood from equation \ref{eq:likelihood}, and a 
measure of the spatial distribution of members would be useful.  The difficulty
is picking the right way to combine these, both functionally and in terms
of relative weight.  More on this later.

\end{document}
