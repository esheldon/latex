%%
%% Beginning of file 'sample.tex'
%%
%% Modified March, 3 2000
%%

% \documentclass{aastex}

% This is what phil used
%\documentclass[manuscript]{aastex}
%\usepackage{aastexug}

%% preprint produces a one-column, single-spaced document:

%\documentclass[preprint]{aastex}

%% preprint2 produces a double-column, single-spaced document:

\documentclass[preprint2]{aastex}

%%
%% Some definitions
%%

\def\rmax{$R_{max}$ }

\slugcomment{Not yet submitted to \apjl}

\begin{document}

\title{A New Method for Estimating Gravitational Shear}

\author{Erin Scott Sheldon}
\affil{Department of Physics, University of Michigan, 500 East University, Ann
Arbor, Mi 48109}
\email{esheldon@umich.edu}

\and

\author{David Johnston}
\affil{University of Chicago}
\email{davej@cubist.uchicago.edu}

\begin{abstract}
We present a new method for estimating gravitational shear which is tuned to
the problem of galaxy-galaxy lensing.  This method uses a simple linear
regression method for estimating the shear of distant source galaxies as a
function of their angular or projected physical separation from foreground 
lenses. An advantage over traditional techniques is that it accounts 
for the clustering of the foreground sample in a natural way without reference
to a measurement of the angular correletion function of the foreground sample.
In this sense, the technique is similar to the method of \cite{SR}
(SR), except that it is not parametric and makes no reference to galaxy 
parameters.

This idea was Dave's.  Here I have extended the discussions to a 
weighted least squares treatment and I have discussed possible advantages and
disadvantages of this technique.
\end{abstract}

\keywords{Galaxies: halos---Gravitational Lensing}

\section{Introduction}

\section{Method}
We will make the usual weak lensing assumptions.  First of all, we assume that 
the ellipticity parameter
averages to zero over many galaxies in the absence of lensing. Furthermore,
we assume we can write the shear as
\begin{equation} \label{shapprox}
2\gamma_T \approx \varepsilon
\end{equation}
where $\varepsilon$ is the ellipticity induced by the effects of lensing and
$\gamma_T$ is the tangential shear relative to the position of the lens.  The
usual approach to measuring the shear is to find foreground lens candidates,
such as low redshift galaxies or clusters, and measure the shapes of background
galaxies in the tangential frame of reference, which yields the shear from Eq.
\ref{shapprox}.

We can approach this measurement in another way.  Consider the case of
galaxy-galaxy lensing, where one seeks the average shear due to an ensemble of
foreground lens galaxies.  Because we assume that the average shape of galaxies
in the absence of lensing is round (the ellipticity vector is zero), then any
residual average ellipticity can be ascribed to lensing.  Due to the linearity 
of the lensing equations in the limit of Eq. \ref{shapprox}, we can write 
the complex ellipticity vector for each background galaxy as a sum of shears 
from each foreground galaxy:
\begin{equation} \label{ellipeqn1}
(e)_k = \sum_{j=1}^{N_{bins}}\sum_{q_j=1}^{N_{fgal}}2\gamma_{q_j}\exp(2i\vartheta_{q,k})
\end{equation}
where k refers to the $k_{th}$ source galaxy, j the $j_{th}$ radial bin, while
$q_j$ loops over all foreground galaxies found in the $j_{th}$ radial bin.  The
angle between foreground and background galaxies is denoted by $\vartheta$.  In
principle, the radial bins must extend to $R=\infty$, but for galaxy surveys
with large sky coverage we can assume that the contribution from foreground
galaxies beyond some \rmax is small, as long as we only examine the shear
profile within some $R_2(R_{max})$.

If we further assume that all foreground galaxies have the same physical
parameters (i.e. we are averaging over them) then Eq. \ref{ellipeqn1} reduces
to
\begin{equation} \label{ellipeqn2}
(e)_k = \sum_{j=1}^{N_{bins}}2\gamma_j\sum_{q_j=1}^{N_{fgal}}\exp(2i\vartheta_{q,k})
\end{equation}
Note that there are really two equations here, one for each element of the
ellipticity parameter.  
The actual shear will be the sum of the shear estimated from $e_1$ and
$e_2$.

The problem has been reduced to finding the average
shear in a finite number of radial bins.  This can be accomplished using
chi-squared minimization. The equation for the $e_1$ component of the ellipticity
vector is:
\begin{equation} \label{e1chisq}
\chi^{2} = \sum_{k=1}^{N_{source}}{(e_{1})_k - \sum_j2\gamma_j\sum_{q_j}\cos(2\vartheta_{q,k}) \over \sigma_k^{2}}
\end{equation}
Eq. \ref{e1chisq} can be minimized with respect to the coefficients
$\gamma_j$, yeilding the usual solution in terms of normal equations:
\begin{equation}
2(\gamma) = (a^{T}a)^{-1}a^{T}(e_{1}) = Cb
\end{equation}
where $C_{k,l}^{-1} = (a^{T}a)_{k,l}$ is the inverse covariance matrix and 
\begin{eqnarray}
(a^{T}a)_{k,l} & = & {\sum_{i=1}^{N_{source}}{S_{k,i}S_{l,i} \over
\sigma_{i}^{2} } } \nonumber \\
S_{k,i} & = & {\sum_{q_k=1}^{N_{lens}}\cos(2\vartheta_{q,i})}  \\
b_k & = & \sum_{i=1}^{N_{source}}{(e_{1})_{i}S_{k,i} \over \sigma_{i}^{2}} \nonumber
\end{eqnarray}
The weights $\sigma_{i}^{2}$ can be chosen to be the quadratic sum of the 
intrinsic ``shape noise'' of the galaxies and the measurement error in the 
shape of each background galaxy: 
$\sigma_{i}^{2} = \sigma_{shape}^{2} + \sigma_{err}^{2}$.
Thus a formal uncertainty can be found from the diagonal elements of
$C_{k,l}$.  However, the uncertainties should really be calculated with
bootstrap resampling of the actual data.

\section{Comparison with Standard Methods}
\subsection{Advantages}
1) The shear due to all foreground galaxies within \rmax is accounted for,
   at least to the degree that the sample is complete (e.g. faint foreground
   objects may not be included in the foreground sample).  Thus, there should 
   be no need to deconvolve the profile from $w(\theta)$

2) $w(\theta)$ deconvolution is correct on average because it uses the average
   radial distribution of neighboring lens galaxies.  However, the regression
   method uses the full two dimensional information for each foreground
   background pair.  It can be shown that, if this angular information averages
   out, the covariance matrix $C_{k,l}$ will be diagonal.  However, this is not
   expected and tests have shown that the terms adjacent to diagonal elements
   of the covariance are 8-10\% in the outer radial bins and cannot be
   neglected.

3) Once the matrices $a^{T}$, $(e)$,  and $a^{T}a$ have been constructed, 
   bootstapping of the data can be performed efficiently.

\subsection{Disadvantages}

1) In principle the radial bins must reach to $R_{max}=\infty$ in order to properly
   account for the shear at large radii.  It is not yet clear how significant
   the resultant boundary terms are when the sum in Eq. \ref{ellipeqn2} is 
   truncated at some \rmax, but, through simulations, one should be able to
   find a radius $R_2(R_{max})$ within which the reconstructed profile is
   correct.

2) All source galaxies must be chosen such that they are \rmax away from the
   image boundary.  For a typical SDSS stripe, this amounts to cutting 
   $\gtrsim 10-12\%$.

\acknowledgments

\begin{thebibliography}{}
\bibitem[Schneider \& Rix(1997)]{SR} Schneider, P., Rix, H., 1997, ApJ, 474, 25
\end{thebibliography}

\end{document}
