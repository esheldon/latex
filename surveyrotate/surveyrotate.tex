%\documentclass{aastex}

%% -- This is what phil used
%\documentclass[manuscript]{aastex}
%\usepackage{aastexug}

%% -- preprint produces a one-column, single-spaced document:
\documentclass[preprint]{aastex}

%% -- preprint2 produces a double-column, single-spaced document:
% \documentclass[preprint2]{aastex}

%% -- EMULATE APJ
%\documentclass{aastex}
%\usepackage{emulateapj5}

\def\eone{e$_1$}
\def\etwo{e$_2$}

\begin{document}

\title{Converting SDSS Shape Measurements from CCD Coordinates to Survey
Coordinates}

\author{Erin Scott Sheldon}

\section{Introduction}
We measure the ellipticity parameters (\eone,\etwo) in CCD coordinates,
(row,col). This coordinate system is fine within a single run, but is not a 
well defined for the survey as a whole.  The logical coordinate system is 
survey coordinates, ($\lambda,\eta$), which is parallel to the center of 
each SDSS stripe.  This coordinate
system is not parallel to the (row,col) system, however, especially in the
outer camera columns and near the ends of stripes where the two coord. systems
can be misaligned by as much as three degrees.

\section{Transforming (\eone, \etwo) Into Survey Coordinates}
The survey will be at least 30$\degr$ from the poles in ($\lambda,\eta$) 
\footnote{see
http://www.astro.princeton.edu/PBOOK/strategy/strategy.htm\#mapProjectionGridFig and\\
http://www-sdss.fnal.gov:8000/$\sim$sdssdp/status/imaging/imagingStatus.gif},
so the misalignment with the (row,col) system will always be small. In fact, 
within a single field, the 
misalignment can accurately be represented by a simple
rotation. Fig. \ref{grid_rotate} illustrates this for the first field in 
run 756, camera column 1, where I have converted a grid of (row,col) points
to the ($\lambda, \eta$) coord. system (using the asTrans file
\footnote{http://www-sdss.fnal.gov:8000/dm/flatFiles/asTrans.html}), and 
then tangent plane projected
in units of pixels. Note this field is outside of the survey, and is a 
worst case senario. Fig. \ref{fieldrot} shows this angle as a function
of field for 756 camcol 6.

Because the transformation to survey coordinates can locally be represented
by a rotation, we can simply rotate $(e)$ = (\eone,\etwo). The ellipticity
parameters (\eone,\etwo) 
transform under a rotaion $\theta$ as $(e)^\prime = R(2\theta)*(e)$. 
Fig. \ref{rote1e2} shows such a rotation for an angle of three degrees.

\section{Conclusions} 
The transformation of (\eone,\etwo) to survey coordinates 
deviates from a simple
rotation by $\lesssim 1\%$, so this should be fine for studies such
as galaxy-galaxy lensing where a systematic leftover distortion in 
a particular direction will just cancel. For cosmic shear, however, 
one might want to apply a more sophisticated transformation.

\begin{figure}
\plotone{grid_rotate.ps}
\figcaption{A grid of (row,col) points in run 756, column 6, field 11,
converted to survey coordinates ($\lambda, \eta$) and tangent plane projected
in units of pixels.
The solid lines are drawn to help illustrate the simple rotation. The
transformation differs from a simple rotation by no more than a few percent
anywhere in the run (within a given field).\label{grid_rotate}}
\end{figure}

\begin{figure}
\plotone{surveyrot_000756_6_1_N3.ps}
\figcaption{The angle between survey coordinages ($\lambda, \eta$) and 
CCD coordinates (row,col) as a function of field for camera column
6 in run 756. Note the ends of the run are actually outside of the
survey area.\label{fieldrot}}
\end{figure}

\begin{figure}
\plotone{test_rotate_e1e2.ps}
\figcaption{Illustration of the rotaion of \eone, \etwo by an
angle of three degrees \label{rote1e2}}
\end{figure}

\end{document}
